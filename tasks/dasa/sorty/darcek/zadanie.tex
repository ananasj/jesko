\input ../../../../include/include.tex

\begin{document}

\velkynadpis{Nepodarený darček}

Ježko Pichliač mal nedávno narodeniny. A ako darček, dostal od svojej mamy postupnosť celých čísiel.
Bohužiaľ, táto postupnosť sa ježkovi hrozne nepáčila. Rozhodol sa ju zmeniť, nechcel však zarmútiť
svoju mamu, preto si povedal, že zmení práve jedno číslo.

Pre každú pozíciu $i$ by ho teraz zaujímalo, aké najmenšie číslo môže byť na $i$-tom mieste, ak v
pôvodnej postupnosti vymení jeden prvok a túto postupnosť utriedi od najmenšieho prvku.

\nadpis{Úloha}

Na vstupe dostanete postupnosť celých čísiel z rozsahu $1$ až $10^9$. Jedno číslo tejto postupnosti
musíte nahradiť iným číslom. Nové číslo musí byť iné ako to staré a musí opäť ležať v intervale $1$
až $10^9$. Pre každú pozíciu vypíšte, aké najmenšie číslo sa môže nachádzať na danej pozícii po
výmene jedného čísla a zoradení postupnosti. 

\nadpis{Vstup}

Na prvom riadku je číslo $n$ ($1 \leq n \leq 10^5$) -- počet čísiel v postupnosti.

Na druhom riadku je $n$ celých čísiel z rozsahu $1$ až $10^9$, určujúce postupnosť.

\nadpis{Výstup}

Na jeden riadok vypíšte $n$ medzerou odelenných čísiel. $i$-te číslo určuje najmenšie číslo, ktoré
vie byť na $i$-tej pozícii.

\nadpis{Príklady}

\vstup
5
4 7 2 1 6
\vystup
1 1 2 4 6
\koniec

\end{document}
