\input ../../../include/include.tex

\begin{document}

\velkynadpis{Závažia}

Ježko má rovnoramenné vahy a veľa závaží rôznej, ale aj rovnakej hmotnosti. Teraz by ich chcel
rozdeliť do vrecúšok aby v nich mal poriadok. A samozrejme, do každého vrecúška môžu ísť len závažia
rovnakej hmotnosti. Zistite, koľko najviac závaží dá do jedného vrecúška.

\nadpis{Úloha}

Máme $n$ závaží, každé má svoju hmotnosť. Zistite najväčší počet závaží s rovnakou váhou a aj danú
váhu.

\nadpis{Vstup}

Na prvom riadku je číslo $n$ ($1 \leq n \leq 10^5$) -- počet závaží.

Na druhom riadku je $n$ celých čísiel v rozmedzí $1$ až $1000$.

\nadpis{Výstup}

Vypíšte dve čísla oddelené medzerou. Prvé predstavuje číslo, ktoré je zastúpené najviackrát a druhé
počet, koľkokrát sa dané číslo vyskytuje. Ak sa vyskytuje najviackrát viac čísiel, vypíšte najmenšie
z nich.

\nadpis{Príklady}

\vstup
3
1 3 2
\vystup
1 1
\komentar
Všetky čísla $1$, $2$ a $3$ sa nachádzajú medzi závažiami raz. Vypíšeme $1$, lebo je z nich troch
najmenšie.
\koniec

\vstup
5
1 5 3 5 4
\vystup
5 2
\komentar
Číslo $5$ sa vyskytuje medzi závažiami $2$.
\koniec

\end{document}
