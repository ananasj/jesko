\input ../../../../include/include.tex

\begin{document}

\velkynadpis{Strašná žúrka}

Včera austronautka Maja organizovala strašnú žúrku. Boli tam všetky dôležité zvieratká z lesa. Ráno
sa Maja zobudila s totálnym oknom. A zistila, že niekto jej prehádzal, jej pekne uložené trofeje zo
súťaží o stavaní mostov.

Hneď chcela ísť hľadať vyníka, ale nie je si istá, či to nespravila sama, však bohvie, čo včera
robila. Je si však istá, že ak to spravila ona, vymenila práve dva prvky. Pomôžte jej zistiť, či to
spravila ona, alebo niekto iný.

\nadpis{Úloha}

Na vstupe dostanete postupnosť celých čísiel. Zistite, či sa z nej dá vyrobiť usporiadaná postupnosť
výmenou najviac dvoch prvkov.

\nadpis{Vstup}

Na prvom riadku je číslo $n$ ($1 \leq n \leq 10^5$) -- počet čísiel v postupnosti.

Na druhom riadku je $n$ celých čísiel z rozsahu $1$ až $10^9$, určujúce postupnosť.

\nadpis{Výstup}

Ak sa dá postupnosť uporiadať pomocou najviac jednej výmeny, vypíšte \texttt{"Maja"} (bez
úvodzoviek). Inak vypíšte \texttt{"neMaja"}.

\nadpis{Príklady}

\vstup
3
5 7 9
\vystup
Maja
\koniec

\vstup
4
2 1 8 12
\vystup
Maja
\koniec

\vstup
4
2 1 12 8
\vystup
neMaja
\koniec

\end{document}
