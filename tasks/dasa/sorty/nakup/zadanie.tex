\input ../../../../include/include.tex

\begin{document}

\velkynadpis{Nákupy so zľavou}

% proofread by Sameth (preklepy, veci na zozname -> zo zoznamu)
% Navyse nerozummiem, ako funguje kombinovanie zliav, asi by bolo lepsie to poriadne rozpisat (je to tak, ze si moze vybrat z_i veci + 0-2 zadarmo, a tych sa uz nemoze tykat zlava?)
Ježko Jano bol minule na nákupoch v jednom nemenovanom obchodnom reťazci. Tam mali veľké množstvo
zliav. Jano si chcel kúpiť $n$ vecí, s cenami $c_1, c_2 \dots c_n$. Každú si môže kúpiť aj normálne,
keď zaplatí jej plnú sumu, alebo ju môže kúpiť v rámci akcie.

Akcie fungujú nasledovne. V obhode ponúkajú $m$ akcií. Ak si chce uplatniť $i$-tu akciu, musí si
kúpiť $z_i$ vecí a k týmto veciam dostane $0$, $1$ alebo $2$ veci zadarmo, presný počet si môže
vybrať. Jano si môže zvoliť, koľko vecí zadarmo si zoberie. Musí však platiť, že veci, ktoré si
zoberie zadarmo, nie sú drahšie ako najlacnejšia z vecí kúpených v tejto akcii. Každú akciu môže
Jano použiť ľubovoľne veľa krát a môže ich ľubovoľne kombinovať, ale, samozrejme, na jednu kupovanú
vec mu nedovolia použiť viacero akcií naraz.

Zistite, koľko najmenej peňazí musí Jano zaplatiť, ak využije zľavový systém optimálne.

\nadpis{Úloha}

Dostanete popis vecí, ktoré si chce Jano kúpiť a popis zliav, ktoré obchod ponúka. Zistite koľko
najmenej musí Jano zaplatiť, ak si chce kúpiť všetky veci zo zoznamu a žiadnu naviac.

\nadpis{Vstup}

Na prvom riadku sú dve čísla $n$ a $m$ ($1 \leq n, m \leq 10^5$) -- počet vecí čo si chce Jano kúpiť
a počet zliav.

Na druhom riadku je $n$ celých čísiel $c_1, c_2 \dots c_n$ ($1 \leq c_i \leq 10^9$)-- ceny predmetov,
ktoré si chce Jano kúpiť.

Ďalších $m$ riadkov popisuje jednotlivé akcie. Každý riadok obsahuje jedno celé číslo $z_i$ ($1 \leq
z_i \leq 10^5$). Číslo $z_i$ určuje, koľko vecí si Jano môže kúpiť tak, aby si mohol uplatniť akciu $i$.

\nadpis{Výstup}

Vypíšte jedno číslo: minimálnu sumu peňazí, ktorí musí Jano zaplatiť za nákup všetkých vecí. Dajte
si pozor, že výsledok sa nemusí zmestiť do normálnej $32$ bitovej premennej.

\nadpis{Príklady}

\vstup
\vystup
\koniec

\end{document}
