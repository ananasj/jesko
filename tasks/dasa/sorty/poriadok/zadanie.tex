\input ../../../../include/include.tex

\begin{document}

\velkynadpis{Poriadok}

% proofread by Sameth (rozdelenie 2. vety)
Malý ježko bol znovu neposlušný, preto ho jeho mama poslala za trest do izby, aby si upratal svoje
hračky. Teraz by chcela ísť skontrolovať, či sú už hračky upratané. Keďže ich je veľa, zišla by
sa jej vaša pomoc.

\nadpis{Úloha}

V ježkovej izbe je $n$ hračiek. Každá z nich má svoju veľkosť $v_i$ vyjadrenú ako kladné celé číslo.
Ježko má hračky upratané, ak ich má buď zoradené od najmenšej po najväčšiu alebo od najväčšej po
najmenšiu. Zistite, či má ježko hračky upratané a podajte o tom správu jeho mame.

\nadpis{Vstup}

Na prvom riadku je číslo $n$ ($1 \leq n \leq 1000$) -- počet ježkových hračiek.

Na druhom riadku je $n$ kladných celých čísel $v_i$ ($1 \leq v_i \leq 1000$) určujúcich veľkosti
ježkových hračiek v poradí, v akom sú zoradené.

\nadpis{Výstup}

Ak sú ježkove hračky upratané, teda sú zoradené buď od najmenšej po najväčšiu, alebo od najväčšej po
najmenšiu, vypíšte jeden riadok so slovom \texttt{poriadok}.

V opačnom prípade vypíšte jeden riadok so slovom \texttt{bordel}.

\nadpis{Príklady}

\vstup
4
1 5 8 13
\vystup
poriadok
\koniec

\vstup
4
5 3 8 1
\vystup
bordel
\koniec

\end{document}
