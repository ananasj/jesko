\input ../../../include/include.tex

\begin{document}

\velkynadpis{Na telesnej výchove}

Učiteľ telesnej si rozdelil svojich žiakov na chlapcov a dievčatá a každú skupinu nechal zoradiť
podľa výšky od najnižšieho po najvyššieho. Keď už boli obe skupinky pripravené na hlásenie, učiteľ
si to rozmyslel a povedal, nech sa podľa výšky zoradia všetci spolu. Zistite ako bude vyzerať rad po
tom ako sa zoradia všetci dokopy.

\nadpis{Úloha}

Na vstupe dostanete dve postupnosti čísiel, ktoré sú zoradené od najmenšieho čísla po najväčšie.
Zistite ako bude vyzerať postupnosť, ktorá vznikne spojením a usporiadaním týchto dvoch postupností.

\nadpis{Vstup}

Na prvom riadku je číslo $n$ ($1 \leq n \leq 10^6$) -- počet prvkov prvej postupnosti.

Na druhom riadku je $n$ kladných celých čísiel $a_i$ ($1 \leq a_i \leq 10^9$) určujúce prvú
postupnosť.

Na treťom riadku je číslo $m$ ($1 \leq m \leq 10^6$) -- počet prvkov druhej postupnosti.

Na štvrtom riadku je $m$ kladných celých čísiel $b_i$ ($1 \leq b_i \leq 10^9$) určujúce druhú
postupnosť.

\nadpis{Výstup}

Vypíšte jeden riadok s $n+m$ medzerami odelenými číslami -- usporiadanú postupnosť, ktorá vznikne
spojením postupností na vstupe.

\nadpis{Príklady}

\vstup
3
5 7 8
4
2 6 7 11
\vystup
2 5 6 7 7 8 11
\koniec

\end{document}
