\input ../../../../include/include.tex

\begin{document}
% a predsa! uloha na precvicenie Queue.
% deti postupne prichadzaju do predsiene.
% raz za cas pride veduci a zavola dalsiu skupinu z K deti.
% ak ich nie je dost, tak vypiseme "malo deti".
% inak tych K deti posleme do lesa a poslednemu (ten, co prisiel najneskor) 
% z tychto deti dame mapu. jeho meno vypiseme.

%gram. korekcia Maja

\velkynadpis{Nočná hra}
Sústredenie KSP, deň štvrtý. Štvrtá nočná hra. Účastníci sa búria, schovávajú 
sa a vedúci ich zúfalo hľadajú. Predsa len sa jedná o hľadanie druhého 
najzápadnejšieho prítoku Nitry! Potrebujú na to teda aj mapu. Vedúci sú už 
natoľko zúfalí, že ich ani neposielajú von podľa pôvodných družiniek: stačí, 
aby sedel počet.

Vedúci sa teda rozbili na dve skupiny: jedna skupina pobehuje po chate, hľadá 
účastníkov a posiela ich do predsiene. Druhá časť je vonku, raz za čas 
vstúpi do predsiene, zoberie $K$ prvých účastnikov (tých, čo tam vstúpili 
skôr), poslednému z nich dá mapu a pošle ich do lesa. Nás zaujímajú mená tých, 
čo dostali mapy (predsa ich potom musia vrátiť!).

\nadpis{Úloha}
Vstup bude obsahovať dva druhy riadkov: mená účastníkov, čo vstupili do 
predsiene, alebo príkaz ''von'' (bez úvodzoviek), ktorý znamená, že prišli 
vedúci zvonku a chcú poslať do lesa ďalšiu skupinu účastníkov. Ak je 
účastníkov v predsieni dosť, tak pošlú von prvých $K$ a treba vypísať meno 
toho, kto dostal mapu. Ak účastníkov dosť nie je (je ich aktuálne v predsieni 
menej ako $K$), tak treba vypísať na výstup riadok ''malo deti'' (bez 
úvodzoviek).
\nadpis{Vstup}
Na prvom riadku vstupu sú celé čísla $N$ a $K$, kde $K$ je veľkosť skupín, v 
ktorých sa účastníci posielajú do lesa ($1 \leq N \leq 10000$, $1 \leq K 
\leq 12$). Ďalších $N$ riadkov obsahuje buď mená účastníkov (obsahujú iba malé 
a velké písmená anglickej abecedy a dĺžka mien neprekročí $12$) alebo riadok 
''von''.
\nadpis{Výstup}
Výstup obsahuje reakcie na každý riadok ''von'' zo vstupu: buď meno účastníka,
ktorý dostal mapu, alebo riadok ''malo deti'' (bez úvodzoviek).
\nadpis{Príklady}

\vstup
4 2
Basa
Zygro
von
Filip
\vystup
Zygro
\komentar
Išli Baša a Zygro, Zygro dostal mapu.
Filip ostal na chate :(
\koniec

\vstup
5 3
Basa
Zygro
von
Filip
von
\vystup
malo deti
Filip
\komentar
Dvaja nestáčili, ale spolu s Filipom už ísť môžu. Mapu dostane Filip.
\koniec

\end{document}
