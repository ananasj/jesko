\input ../../../include/include.tex

\begin{document}

\velkynadpis{Trpaslíci}

% proofread by Sameth
Občas sa stane, že sa na ceste z bane nejaký trpaslík stratí a domov ich príde menej ako $7$. Pre
Snehulienku je však ťažké zistiť, kto sa vlastne stratil, lebo všetci vyzerajú rovnako -- špicatý
klobúk je vlastne jediné čo Snehulienka z tej výšky vidí.

Našťastie, nie všetci trpaslíci majú rovnakú váhu. Ak sa jej teda domov vráti menej ako $7$
trpaslíkov postaví ich všetkých spolu na váhu a častokrát už len z tohto vie, kto sa stratil. Viete
to aj vy?

\nadpis{Úloha}

Dostanete hodnoty, koľko váži ktorý z trpaslíkov. Taktiež sa dozviete, koľkí sa vrátili domov a aká
bola ich spoločná váha. O každom trpaslíkovi zistite, či je doma, chýba, alebo sa nedá povedať, kde
je.

\nadpis{Vstup}

Na prvom riadku je sedem celých čísel menších ako $200$ -- váhy jednotlivých trpaslíkov.

Na druhom riadku sú dve čísla $n$ ($0 \leq n < 7$) a $h$. $n$ určuje počet trpaslíkov, ktorý sa
vrátili domov a $h$ ich spoločnú váhu. Môžete predpokladať, že $h$ naozaj vzniklo ako súčet
niektorých $n$ čísel z prvého riadku.

\nadpis{Výstup}

Vypíšte $7$ riadkov, na každom bude číslo trpaslíka a jeden z jeho možných stavov: \texttt{doma},
\texttt{chyba}, \texttt{nevedno}. Dodržte formát ako vo vzorových príkladoch.

\nadpis{Príklady}

\vstup
10 11 12 13 14 16 15
6 75
\vystup
1 doma
2 doma
3 doma
4 doma
5 doma
6 chyba
7 doma
\koniec

\vstup
10 11 12 13 14 16 16
6 76
\vystup
1 doma
2 doma
3 doma
4 doma
5 doma
6 nevedno
7 nevedno
\koniec

\end{document}
