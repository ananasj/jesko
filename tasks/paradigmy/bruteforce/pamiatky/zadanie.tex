\input ../../../../include/include.tex

\begin{document}

\velkynadpis{Prehliadka pamiatok}

Ježko Pichliač ide na dovolenku do exotickej krajiny -- Aburdistanu. V Aburdistane je $n$ svetovo
preslávených pamiatok a ježko by ich chcel postupne navštíviť všetky. O každej pamiatke vie, koľko
radosti z výletu mu dané pamiatka pridá keď ju navštívi. Toto číslo môže byť aj záporné.

Ježko si chce teraz naplánovať svoj výlet. Nezáleží mu v akom poradí navštívi pamiatky, chce však
každú navštíviť práve raz a chce, aby v žiadnom momente neklesla jeho radosť z výletu pod $0$. Na
začiatku má samozrejme radosť $0$. Pomôžte mu zistiť, koľkými rôznymi spôsobmi si vie naplánovať
svoj výlet.

\nadpis{Úloha}

Na vstupe dostanete počet pamiatok $n$ a tiež $n$ čísiel $r_i$ -- množstvo radosti, ktoré nám pridá
návšteva pamiatky $i$. Zistite počet rôznych spôsobov návštevy všetkých pamiatok tak, aby v žiadnom
momente nebola ježkova radosť záporná.

\nadpis{Vstup}

Na prvom riadku je číslo $n$ ($1 \leq n \leq 9$) -- počet pamiatok v Aburdistane.

Na druhom riadku je $n$ celých čísiel $r_i$ ($-1000 \leq r_i \leq 1000$) -- množstvo radosti
získanej pri návšteve $i$-tej pamiatky.

\nadpis{Výstup}

Vypíšte jedno celé číslo -- počet rôznych možností.

\nadpis{Príklady}

\vstup
3
1 -2 2
\vystup
3
\komentar
Ježko nemôže začať návštevou druhej pamiatky, lebo jeho radosť by hneď klesla pod $0$. Ak začne
prvou pamiatkou, ako ďalšia musí nasledovať tretia, takže máme jednu možnosť $(1, 3, 2)$. Ak začne
treťou, obe zvyšné možnosti sú priateľné, teda $(3, 1, 2)$ a $(3, 2, 1)$. Dokopy máme $3$ rôzne
výlety.
\koniec

\vstup
5
-540 817 331 366 -253
\vystup
50
\koniec

\end{document}
