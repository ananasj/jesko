\input ../../../../include/include.tex

\begin{document}

\velkynadpis{Permutácia čísiel}

Na vstupe dostanete $n$ čísiel, každé má presne $k$ cifier. Nájdite takú permutáciu cifier v týchto
číslach, že rozdiel najväčšieho a najmenšieho takto spermutovaného čísla bude čo najmenší. Všimnite
si, že na všetky čísla má byť použitá tá istá permutácia a takisto, že dovoľujeme čísla, ktoré
začínajú jednou alebo viacerými nulami.

\nadpis{Vstup}

Na prvom riadku sú dve čísla $n$ a $k$ ($1 \leq n, k \leq 8$) -- počet čísiel na vstupe a počet
cifier v nich. Nasleduje $n$ $k$-ciferných čísiel, niektoré môžu začínať prebytočnými nulami.

\nadpis{Výstup}

Vypíšte najmenší možný rozdiel najväčšieho a najmenšieho čísla, ktoré vzniknú permutáciou cifier v
pôvodných číslach.

\nadpis{Príklady}

\vstup
3 3
010
909
012
\vystup
3
\komentar
Ak použijem permutáciu $(2, 1, 3)$ dostanem čísla $100$, $099$ a $102$, takže rozdiel najväčšieho a
najmenšieho čísla bude $102 - 99 = 3$.
\koniec

\vstup
6 4
5237
2753
7523
5723
5327
2537
\vystup
2700
\koniec

\end{document}
