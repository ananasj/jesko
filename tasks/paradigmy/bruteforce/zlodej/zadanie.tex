\input ../../../../include/include.tex

\begin{document}

\velkynadpis{Zlodej}

Zo svojej ultraľahkej kuše ste vystrelili šíp s hákom, za ktorým sa tiahlo tenké, no pevné lano. Hák
sa zachytil na rebríku neďalekej budovy. Potichu ako mačka prerúčkujete cez lano, ktoré je teraz
vo výške dvadsiatich metrov. Nepozeráte sa dole, svoj cieľ máte pred sebou.

Zoskočíte na balkón prominentného milionára. Vytiahnete vašu najnovšiu hračku, laserový rezač skla,
pomocou, ktorého spravíte dieru do sklennej výplne. Opatrne si otvoríte dvere a potichu sa vtiahnete
dnu. Do vzduchu pred sebou nastriekate aerosolový sprej. Hneď sa pred vami objaví slabo svietiaca
červená sieť poplašného zariadenia.

V štýle profesionálneho gymastu sa prehýbaš medzi červenými lúčmi až nakoniec skončíš uprostred
obývačky. V obývačke je $n$ rôznych vecí. Je ti jedno, ktoré z nich zoberieš, lebo všetky majú
obrovskú cenu. Každý má však nejakú hmotnosť a batoh, ktorý máš na chrbte unesie len $w$ váhy. A ty
chceš svoj batoh naplniť čo najviac. Aká je najväčšia hmotnosť nepresahujúca $w$, ktorú vieš zbaliť?

\nadpis{Úloha}

Na vstupe dostanete číslo $n$ a $w$ -- počet predmetov a váhu, ktorú unesie batoh. Pre každú z vecí
dostanete hodnotu $v_i$ -- hmotnosť $i$-teho objektu. Zistite, akú najväčšiu hmotnosť viete nabaliť
do batoha s nosnosťou $w$ použitím daných predmetov.

\nadpis{Vstup}

Prvý riadok obsahuje dve čísla $n$ a $w$ ($1 \leq n \leq 25$, $1\leq w \leq 10^9$) -- počet vecí a
nostnosť batoha.

Druhý riadok obsahuje $n$ celých čísiel $v_1$ až $v_n$ ($1 \leq v_i \leq 10\,000 $) -- váhy
jednotlivých vecí.

\nadpis{Výstup}

Jedno celé číslo -- maximálny počet vecí, ktoré si vieš zbaliť do batoha vzhľadom na jeho obmedzenú
hmotnosť.

\nadpis{Príklady}

\vstup
3 5
5 2 2
\vystup
5
\komentar
Použitím dvoch vecí s váhou $2$ dosiahnem len objem $4$. Iné možnosti presiahnu nosnosť batoha.
\koniec

\end{document}
