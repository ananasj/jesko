\input ../../../include/include.tex

\begin{document}

\velkynadpis{Faktoriál}

Na vstupe dostanete jedno celé číslo $k$. Nájdite najmenšie $x$ také, že $x!$ má v desiatkovom
zápise na konci aspoň $k$ núl.

\nadpis{Vstup}

Jedno celé číslo $k$ ($1 \leq k \leq 10^7$).

\nadpis{Výstup}

Jedno celé číslo $x$ -- najmenšie číslo, ktorého faktoriál má na konci aspoň $k$ núl.

Môžete predpokladať, že $x$ sa vám v pohode zmestí do $32$ bitovej premennej so znamienkom -- teda
normálny \texttt{int}.

\nadpis{Príklady}

\vstup
1
\vystup
5
\komentar
Prvých niekoľko faktoriálov: $1$, $2$, $6$, $24$, $120$. Prvý faktoriál, ktorý má na konci $0$ je
$5!$ odpoveď je teda $5$.
\koniec

\vstup
5
\vystup
25
\koniec

\vstup
6
\vystup
25
\komentar
Všimnite si, že posledné dva vstupy dávajú rovnaký výstup. Je to preto, že čísla menšie ako $25$
majú na konci najviac $4$ nuly, ale číslo $25!$ ich má na konci rovno $6$.
\koniec

\end{document}
