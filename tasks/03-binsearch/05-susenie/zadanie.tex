\input ../../../include/include.tex

\begin{document}

\velkynadpis{Sušenie oblečenia}

Práve ste sa vrátili z výletu s množstvom špinavého oblečenia. Hneď ste ho všetko nahádzali do
pračky a oprali. Teraz však prichádza problém. Za chvíľu odchádzate na ďalší výlet a musíte teda čo
najrýchlejšie usušiť všetky mokré veci.

Sušenie funguje nasledovne. Každej veci sa dá priradiť jej aktuálna vlhkosť $v_i$. Ak bude vlhkosť
nulová, vec sa pokladá za vysušenú a ďalej neschne (lebo nemá z čoho). Ak je nejaká vlhká vec
na vzduchu, každú minútu stratí jednu vlhkosť. Našťastie však máte jeden radiátor, na ktorý sa
zmestí práve jeden kus oblečenia. Ak je vec položená na radiátore, každú minútu stratí $k$ vlhkosti.
Ak by v nejakom momente mala vlhkosť menšiu ako $k$ tak uschne úplne.

Veci na radiátore viete každú minútu vymeniť a nezožerie vám to žiaden čas. Zistite, ako
najrýchlejšie viete usišiť všetky mokré veci ak budete používať radiátor optimálne.

\nadpis{Úloha}

Na vstupe dostanete popis vlhkostí všetkých vecí a parameter $k$ radiátora. Zistite najmenší čas
potrebný na úplné usušenie všetkých vecí.

\nadpis{Vstup}

Na prvom riadku je číslo $n$ ($1 \leq n \leq 10^5$) -- počet mokrých vecí.

Na druhom riadku je $n$ celých čísiel $v_i$ ($1 \leq v_i \leq 10^9$) -- vlhkosti jednotlivých vecí.

Posledný riadok obsahuje číslo $k$ ($1 \leq k \leq 10^9$) -- množstvo vlhkosti, ktoré vec stratí
počas jednej minúty na radiátore.

\nadpis{Výstup}

Jedno celé číslo -- najmenší počet minút, za ktoré vieme usušiť všetky kusy oblečenia.

\medskip

\textbf{Upozornenie:} Dávajte si pozor, na úspešné riešenie je treba správne použiť $64$ bitovú
premennú -- \texttt{long long}.

\nadpis{Príklady}

\vstup
3
2 3 9
5
\vystup
3
\komentar
Veci s vlhkosťou $2$ a $3$ necháme uschnúť normálne, vec s vlhkosťou $9$ dáme na radiátor. Všimnite
si, že za $2$ minúty všetko usušiť neviem.
\koniec

\vstup
3
2 3 6
5
\vystup
2
\komentar
Prvú minútu dáme na radiátor vec s vlhkosťou $3$. Po tejto minúte budú mať veci vlhkosti $1$, $0$ a
$5$. Druhú minútu položíme na radiátor tretiu vec, ktorá tým pádom vyschne a prvá vec vyschne za
druhú minútu sama od seba.
\koniec

\end{document}
