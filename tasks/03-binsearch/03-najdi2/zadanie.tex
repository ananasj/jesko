%gram. kontrola Maja
\input ../../../include/include.tex

\begin{document}

\velkynadpis{Nájdi číslo 2}

Na vstupe máte usporiadanú postupnosť $n$ kladných aj záporných čísel. Vašou úlohou je odpovedať na
$q$ otázok. Každá otázka je jedno kladné číslo $x$ a pýta sa, či existuje v našej postupnosti také
číslo, ktorého absolútna hodnota je rovná číslu $x$.

\textbf{Upozornenie:} Počet prvkov v postupnosti je dosť veľký, preto usporiadanie tejto postupnosti
podľa absolútnych hodnôt môže byť dosť pomalé. Skúste túto úlohu vyriešiť bez toho, aby ste
zasahovali do vstupnej postupnosti.

\nadpis{Vstup}

Na prvom riadku je číslo $n$ ($1 \leq n \leq 5\cdot 10^6$) -- počet čísel v postupnosti.
Na druhom riadku je $n$ rôznych celých čísel z rozsahu $-10^9$ až $10^9$ určujúce postupnosť.

Nasleduje číslo $q$ ($1 \leq q \leq 1000$) -- počet otázok. Posledných $q$ riadkov určuje jednotlivé
otázky. Všetky otázky sú v rozsahu $0$ až $10^9$.

\nadpis{Výstup}

Na každú otázku vypíšte na samostatný riadok odpoveď -- písmeno \texttt{A}, ak sa vo vstupnej
postupnosti nachádza číslo, ktorého absolútna hodnota sa rovná $x$, inak vypíšte \texttt{N}. 

\nadpis{Príklady}

\vstup
5
-5 -3 1 7 88
3
2
7
5
\vystup
N
A
A
\komentar
V poli sa nenachádza $2$ ani $-2$, takže prvá odpoveď je \texttt{N}. V druhej otázke sa nachádza
číslo $7$, v tretej zase $-5$.
\koniec

\end{document}
