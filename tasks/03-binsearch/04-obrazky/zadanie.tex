\input ../../../include/include.tex

\begin{document}

\velkynadpis{Obrázkový hlavolam}

Ježko Pichliač dostal na narodeniny zvláštny obrázkový hlavolam. Skladá sa z prázdnej mriežky
$n\times n$ a $k$ sád obrázkov. Každá sada obsahuje nejaký počet rovnakých obrázkov, navzájom medzi
sadami sú však obrázky rôzne a dokopy ich je presne $n^2$.

Pichliačovou úlohou je vložiť obrázky do mriežky tak, aby vytvoril čo najviac rovnakých riadkov. Ak
má napríklad mriežku veľkosti $3$ a má $4$ hviezdičky, $2$ pluská, $2$ bodky a $1$ rovná sa, jeho
mriežka môže vyzerať takto:

\begin{verbatim}
* * +
. * *
+ = .
\end{verbatim}

V tomto prípade má však len jeden rovnaký riadok, lebo všetky vyzerajú rôzne. Ak by obrázky však
preusporiadal nasledovným spôsobom, dostal by $2$ rovnaké riadky, čo je zároveň aj najlepšie čo vie
dosiahnuť.

\begin{verbatim}
* * .
* * .
+ + =
\end{verbatim}

Pomôžte mu zistiť, koľko najviac rovnakých riadkov vie mať s danou sadou obrázkov.

\nadpis{Úloha}

Na vstupe dostanete veľkosť mriežky, počet rôznych sád obrázkov a veľkosti jednotlivých sád. Zistite
koľko najviac rovnakých riadkov vie ješko z týchto obrázov vyskladať.

\nadpis{Vstup}

Na prvom riadku je číslo $n$ a $k$ ($1\leq n \leq 40000$, $1 \leq k \leq 50000$) -- veľkosť mriežky
a počet sád obrázkov.

Nasleduje $k$ riadkov, každý obsahuje jedno celé číslo medzi $1$ a $n^2$ určujúce počet obrázkov v
príslušnej sade. Súčet týchto čísiel je presne $n^2$.

\nadpis{Výstup}

Vypíšte jedno celé číslo -- najväčší možný počet rovnakých riadkov, ktoré vieme vytvoriť ukladaním
obrázkov do mriežky.

\nadpis{Príklady}

\vstup
3 4
4
2
1
2
\vystup
2
\komentar
Toto je príklad zo zadania.
\koniec

\end{document}
