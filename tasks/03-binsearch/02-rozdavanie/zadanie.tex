%gram. kontrola Maja

\input ../../../include/include.tex

\begin{document}

\velkynadpis{Rozdávanie cukríkov}

Prišiel Mikuláš a priniesol vám veľa cukríkov. A teraz ich chce všetky rozdať. Nie všetci si však
zaslúžite rovnako veľa cukríkov. Mikuláš o každom vie, koľko najviac cukríkov si zaslúži a určite mu
nikdy nedá viac, aj keď môže dať menej. Chcel by však byť spravodlivý a rozdať cukríky čo
najrovnomernejšie. Vo vlastnom záujme mu pomôžte.

\nadpis{Úloha}

Na vstupe dostanete počet cukríkov, ktoré chce Mikuláš rozdať a taktiež čísla $a_i$ určujúce, že
človek $i$ si nezaslúži viac ako $a_i$ cukríkov. Nájdite najmenšie $x$ také, že Mikuláš rozdá všetky
cukríky a nikto nedostane viac ako $x$ cukríkov.

\nadpis{Vstup}

Na prvom riadku sú čísla $n$ a $p$ ($1 \leq n \leq 1000$, $1 \leq p \leq 10^9$) -- počet ľudí,
ktorým chce Mikuláš cukríky rozdať a celkový počet cukríkov.

Nasleduje $n$ čísel $a_i$ ($1 \leq a_i \leq 10^6$) -- maximálny počet cukríkov, ktoré si zaslúži
$i$-ty človek.

\nadpis{Výstup}

Vypíšte najmenšie $x$ také, že Mikuláš vie rozdať všetky cukríky a nikto nedostane viac ako $x$
cukríkov. Ak také $x$ neexistuje, vypíšte \texttt{Nic nedostanete}

\nadpis{Príklady}

\vstup
5 34
9 8 9 9 4
\vystup
8
\komentar
Prvým dvom dá Mikuláš $8$ cukríkov, ďalším dvom len $7$ a poslednému $4$, lebo si viac nezaslúži.
Takto rozdá všetkých $34$ cukríkov. Všimnite si, že ak by chcel dať najviac $7$ cukríkov, tak ich
rozdá len $32$, čo je málo.
\koniec

\vstup
3 7
1 1 4
\vystup
Nic nedostanete
\komentar
Aj keby dal každému plný počet cukríkov, rozdá ich len $6$.
\koniec

\end{document}
