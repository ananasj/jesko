%gram. oprava Maja

\input ../../../include/include.tex

\begin{document}

\velkynadpis{Nájdi číslo}

Na vstupe máte usporiadanú (od najmenšieho po najväčšie) postupnosť $n$ prvkov. Ďalej nasleduje $q$
otázok, každá sa pýta, na ktorej pozícii sa v tomto poli nachádza číslo $x$. Vypíšte túto pozíciu
alebo $-1$, ak sa tam číslo nenachádza.

\nadpis{Vstup}

Na prvom riadku je číslo $n$ ($1 \leq n \leq 10^5$) -- počet čísel v postupnosti.
Na druhom riadku je $n$ rôznych celých čísel z rozsahu $1$ až $10^9$ určujúce postupnosť.

Nasleduje číslo $q$ ($1 \leq q \leq 10^5$) -- počet otázok. Posledných $q$ riadkov určuje jednotlivé
otázky. Všetky otázky sú opäť v rozsahu $1$ až $10^9$.

\nadpis{Výstup}

Na každú otázku vypíšte na samostatný riadok odpoveď -- pozíciu, kde sa dané číslo nachádza vo
vstupnom poli alebo $-1$, ak sa nenachádza. Pozície sú číslované od $0$ po $n-1$.

\nadpis{Príklady}

\vstup
5
1 2 8 9 15
3
2
5
15
\vystup
1
-1
4
\koniec

\end{document}
