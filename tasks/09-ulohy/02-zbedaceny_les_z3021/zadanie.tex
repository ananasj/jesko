\input ../../../include/include.tex

\begin{document}

\velkynadpis{Zbedačený les}

Včera sa bol zajac poprechádzať po lesíku. Len tak si hopkal, keď sa zrazu spoza
brezy vyvalil nákladiak. Zajac sa rozhodol ho nasledovať. Dobehol ho až vtedy, keď už nákladiak
vykladal kopu sladkostí doprostred davu čakajúcich zvieratiek.
Tie sa na sladkosti vrhli tak rýchlo, ako im to len papuľky a bruchá dovolili.
Obaly šušťali, jazyky mľaskali, na papuľkách sa rozmazávala čokoláda\dots\ no, vyzeralo to
tam dosť negustiózne.

Už dlhé roky to v našich lesoch takto vyzerá. Odkedy sa zvieratká naučili objednávať si donášku
potravy z fastfoodov a obchodných reťazcov, len žerú a priberajú.

\nadpis{Úloha}

Kráľom lesa je najstaršie zo všetkých zvieratiek. Chceli by ste zistiť, ktoré to je.
Nepoznáte ale vek jednotlivých zvieratiek.
(Môžete len predpokladať, že všetci obyvatelia lesa majú navzájom rôzny vek, a teda je kráľ
jednoznačne určený.)

Našťastie viete, ako zvieratká tučnejú:
Čím je zvieratko staršie, tým je tučnejšie.
Všetky zvieratká rovnakého druhu pritom tučnejú rovnako rýchlo. Ak teda stretneme napríklad
dva zajace, ten tučnejší z nich je zároveň aj starší. Neviete však vôbec porovnávať zvieratká rôznych
druhov -- napríklad aj chudo vyzerajúci ježko môže byť starší ako tučnučká zebra, ktorá sa práve
skotúľala dole kopcom a premýšľa, ako sa dostať späť.

Na vstupe dostanete o každom zvieratku dva údaje: akého je druhu a aký má obvod pása (teda číslo, udávajúce, ako tučné je).
Zistite, koľko rôznych zvieratiek môže byť na základe týchto údajov kráľom lesa.



\nadpis{Formát vstupu}

V prvom riadku vstupu bude celé číslo $n$ ($1\leq n\leq 10^6$), udávajúce počet zvieratiek.
Aby sa vám ľahšie spracúvalo údaje o zvieratkách, jednotlivé druhy sme očíslovali prirodzenými
číslami z rozsahu od 1 po $n$.

Nasleduje pre každé zvieratko jeden riadok obsahujúci dve medzerou oddelené celé čísla:
najskôr jeho druh $d_i$ ($1\leq d_i\leq n$), potom jeho obvod pása $t_i$ ($1\leq t_i\leq 10^9$).
Môžete predpokladať, že žiadne dve zvieratká rovnakého druhu nemajú rovnaký obvod pása.

\nadpis{Formát výstupu}

Vypíšte jeden riadok a v ňom počet kandidátov na najstaršie zvieratko.

\nadpis{Príklady}

\vstup
5
3 8
2 1000
3 47
2 42
3 123
\vystup
2
\komentar
Nevieme, či je staršie zvieratko druhu $2$ s obvodom pása $1000$ alebo zvieratko druhu $3$ s obvodom
pása $123$. Žiadne zo zvyšných zvieratiek kráľom lesa byť nemôže.
\koniec

\end{document}
