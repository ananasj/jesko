\input ../../../include/include.tex

\begin{document}

\velkynadpis{Zradný rad na obed}

Ako všetci viete, na intrákoch sú dve jedálne, jedna vyššie a druhá nižšie.
Ale v tej dolnej nemajú ani kofolu! Preto všetci matfyzáci musia chodiť hore kopcom až
do hornej jedálne, aby si mohli dať svoj obľúbený nápoj. A keď idú všetci do tej
istej jedálne, vznikajú tam nekonečné rady\dots

Raz sa aj náš hladný Luxusko ocitol v takomto rade na obed. Dlho čakal, aby utíšil
hlasné škvŕkanie v svojom bruchu. Ale keď už bol takmer pred pokladňou, prišiel vedúci jedálne.
Videl, že len preto, aby sa skôr najedli, sa tam všetci predbiehajú, strkajú a bijú.
Preto sa rozhodol, že to takto ďalej nepôjde a vydal nové pravidlo:
študenti sa musia postaviť do radu odznova -- ale už podľa abecedy.

Luxusko už od hladu nevidí a tobôž nemá síl hľadať si teraz nové miesto,
kam sa má postaviť. Našťastie ho však napadlo, ako si to uľahčiť. Rozhodol sa, že
vám zavolá a nadiktuje vám mená všetkých v rade. No a vy mu len poviete, o koľko a ktorým smerom sa má
posunúť, aby v novom rade (tom usporiadanom podľa abecedy) stál na správnom mieste.

\nadpis{Úloha}

Na vstupe dostanete postupnosť ľudí v poradí, v akom teraz stoja v rade.
Jeden z týchto ľudí bude Luxusko. Vašou úlohou bude vypísať, ktorým smerom a o
koľko sa Luxusko posunie oproti svojej súčasnej pozícii, ak sa rad usporiada
podľa abecedy (pričom prvý podľa abecedy bude na začiatku radu).

\nadpis{Formát vstupu}

V prvom riadku vstupu bude číslo $n$ ($1\leq n\leq 100000$) udávajúce počet ľudí v
rade. Ďalej bude nasledovať $n$ riadkov, každý obsahujúci jedno meno.
Mená budú tvorené 1 až 47 písmenami anglickej abecedy, pričom iba prvé písmeno
bude veľké. Všetky mená budú navzájom rôzne a práve jedno z nich bude "\texttt{Luxusko}".

\nadpis{Formát výstupu}

Vypíšte jeden riadok, ktorý obsahuje buď "\texttt{o $x$ dopredu}", "\texttt{o
$x$ dozadu}" (pričom $x$ označuje o koľko miest sa Luxusko daným smerom posunie) alebo
"\texttt{neposunie sa}".

\nadpis{Príklady}

\vstup
7
Kaja
Luxusko
Katka
Baska
Maja
Sysel
Olivia
\vystup
o 2 dozadu
\komentar
Po usporiadaní bude v rade postupne Baška, Kaja, Katka, Luxusko, Maja, Olívia a Syseľ.
\koniec

\vstup
3
Pinkofilka
Luxusko
Hihihi
\vystup
neposunie sa
\komentar
Zvyšní dvaja ľudia si svoje miesta v rade vymenia, no Luxusko sa nemusí ani pohnúť.
\koniec

\end{document}
