\input ../../../include/include.tex

\begin{document}

\velkynadpis{Žiariaci gitaristi}

Kto chce byť naozaj dobrý v hre na gitare, ten rýchlo zistí, že príprava a
premyslenosť hrania sú veľmi dôležité.
Jednou z prvých
vecí, ktoré vás na konzervatóriu naučia, je nerobiť zbytočné pohyby. Načo púšťať
% Prvú vec, čo vás na konzervatóriu naučia, je sa pri hraní správne ksichtiť. [tommy]
prst, keď o chvíľu hrá znovu? Načo držať strunu, keď na ňu vôbec nehrám? Tretí
príklad si domyslite sami.

Toto všetko vedia aj na Černobyľskom konzervatóriu. A vedzte, že tam vychovávajú
tých najlepších gitaristov. Majú totiž jednu výhodu -- prsty im rastú rýchlejšie ako
ich stíhajú používať, a preto im nerobí problém chytiť aj tie najkrkolomnejšie
akordy. Taký černobyľský gitarista, verte-neverte, má z praktického hľadiska
nekonečný počet prstov.

\nadpis{Úloha}

Vašou úlohou je pre zadanú postupnosť tónov vypočítať, koľko najmenej pohybov
prstov musí žiak vykonať, aby skladbu zahral. Za jeden pohyb počítame
stlačenie alebo uvoľnenie struny na niektorom pražci (teda stlačenie a následné
uvoľnenie sú dva pohyby).

Tóny sú presne popísané číslom struny a pražca, na ktorom treba danú strunu
stlačiť.(
Mnohí z vás asi vedia, že ten istý tón možno v skutočnosti vylúdiť na viacerých
strunách. Černobyľskí učitelia sú ale mimoriadne tvrdohlaví a prísni, preto ich
študenti ani len neskúšajú použiť inú ako zadanú strunu).
Ak je jedna struna stlačená na viacerých pražcoch naraz, po brnknutí na strunu
zaznie len najvyšší z tónov (ten na pražci s najvyšším číslom).

Na začiatku hrania gitarista nestláča ani jednu strunu. Po skončení môže stláčať
nejaké struny.

\vspace{-0.5em}
\nadpis{Formát vstupu}

V prvom riadku je číslo $n$ ($1 \leq n \leq 1\,000\,000$), ktoré udáva počet
tónov v skladbe. Nasleduje $n$ riadkov, v $i$-tom riadku sú dve čísla $a_i \; b_i$
($1 \leq a_i \leq 6$, $1 \leq b_i \leq 10^9$),
ktoré hovoria, že treba stlačiť strunu $a_i$ na $b_i$-tom pražci.

\vspace{-0.5em}
\nadpis{Formát výstupu}

Vypíšte jedno číslo -- najmenší počet pohybov, ktoré je nutné vykonať na
zahranie skladby.

\nadpis{Príklady}

\vstup
6
1 3
1 4
1 3
2 6
2 6
1 2
\vystup
6
\komentar
Stlačíme 1/3, potom 1/4, potom pustíme 1/4, pustíme 1/3, stlačíme 2/6 a stlačíme
1/2.
\koniec

\end{document}
