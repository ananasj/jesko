\input ../../../include/include.tex

\begin{document}

\velkynadpis{Zahrajme si!}

V KSPárni je mnoho vecí a medzi nimi má svoje čestné miesto aj KSPácky synťák.
Má už svoje roky a naraz zvláda zahrať len jeden tón. Jano sa rozhodol,
že preň spíše zbierku skladieb, a zaumienil si, že tieto skladby sa budú
hrať len jednou (pravou) rukou. Jedna z otázok, ktorú Jano pri písaní skladieb rieši,
je určenie ich náročnosti.

Aby Jano mohol dať svojej tvorivosti čo najväčší priestor, zišiel by sa mu na
určenie náročnosti skladby nejaký program, ktorý by to robil za neho.

\nadpis{Úloha}

Synťák má $k$ klávesov očíslovaných 1 až $k$ zľava doprava.
Hráčova pravá ruka je široká ako $\ell$ klávesov. Pokiaľ má teda hráč palec na $i$-tom klávese, dokáže stlačiť len klávesy
na pozíciách $i$, $i+1$, \dots, $i+\ell-1$. Ak by chcel hráč zahrať iný kláves, musí pohnúť rukou.
Na začiatku skladby je ruka vždy položená na klávesoch 1 až $\ell$.

Počas hrania skladby je možné kedykoľvek pohnúť rukou do ľubovoľnej strany o ľubovoľný počet klávesov.
Súčet vzdialeností, o ktoré sa ruka pohla počas hrania, nazvime \emph{náročnosť zahrania}.
\emph{Náročnosť skladby} je potom najmenšia \emph{náročnosť zahrania} pre všetky možné spôsoby, akými sa dá skladba zahrať.

%% este raz zhrnutie co maju robit
Napíšte program, ktorý pre danú skladbu zistí jej náročnosť, teda najmenšiu vzdialenosť, o ktorú sa musí
dokopy pohnúť ruka počas jej hrania.


\nadpis{Formát vstupu}

Na prvom riadku sú medzerami oddelené čísla $n$, $k$ a $\ell$, pričom $1\leq n\leq 500\,000$ a
$1\leq \ell\leq k\leq 10^9$.

Nasleduje $n$ riadkov s notami
Janovej skladby v poradí, v akom ich treba zahrať. Nota je určená číslom klávesu,
ktorý treba na KSPáckom synťáku stlačiť, čiže sú to celé čísla v rozsahu 1 až $k$.  

\nadpis{Formát výstupu}

Na výstupe bude jedno celé číslo -- celková vzdialenosť, ktorú musí ruka prejsť na zahranie
skladby na vstupe. Uvedomte si, že výsledok sa nemusí vždy zmestiť do 32-bitovej premennej.

\nadpis{Príklady}

\vstup
5 8 4
1
2
3
4
5
\vystup
1
\komentar
Celá skladba sa nedá zahrať bez pohybu. Na jeden pohyb to však už ide --
napríklad po prvom tóne posunieme ruku o jeden kláves doprava.
\koniec

\vstup
3 5 1
1
5
3
\vystup
6

\komentar
Po prvom tóne posunieme ručičku o 4 klávesy doprava, po druhom tóne o 2 klávesy
späť.

\koniec

\end{document}
