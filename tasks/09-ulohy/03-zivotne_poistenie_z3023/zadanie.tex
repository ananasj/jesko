\input ../../../include/include.tex

\begin{document}

\velkynadpis{Životné poistenie}

Akiste ste už počuli, že policajti sú v najvyššom ohrození života práve deň pred
odchodom do dôchodku. Mnohí si preto pobalia veci a rozlúčia sa s kolegami o deň
či dva skôr. No inšpektor Kloten vie svoje -- takýto triviálny manéver nedokáže
oklamať zlomyseľný osud.

Kloten si všimol, že najdôležitejšie je zďaleka sa vyhýbať situáciám, ktoré by
mohol osud považovať za ironické a pobaviť sa na nich. Jeho riešenie je preto to
najnudnejšie možné -- spoľahol sa na životné poistenie. Inšpektor uzavrel
niekoľko zmlúv tak, aby bol poistený počas každého z $n$ dní zostávajúcich do
jeho dôchodku. Dokonca väčšinou bude poistený viacerými zmluvami naraz.

Teraz mu však v hlave skrsla hrozivá myšlienka: Čo ak sa niekedy stane, že
v jeden deň bude poistený trebárs štrnástimi zmluvami naraz, a zrazu nasledujúci deň
už len tromi?
% misof: tu povodne bolo "je počas niektorého dňa poistený výrazne menším počtom zmlúv ako v okolité dni?" co je menej presne
Nebola by to irónia osudu, keby sa mu stala smrteľná nehoda práve v taký deň?%


\nadpis{Úloha}

Do dôchodku zostáva $n$ dní. Počas $i$-teho z nich už je inšpektor Kloten poistený
$a_i$ zmluvami. Môže uzavrieť ľubovoľný počet ďalších zmlúv -- pre každú novú
zmluvu si môže ľubovoľne určiť interval dní $[x, y]$, počas ktorých má byť platná ($1 \leq x \leq y
\leq n$). Kloten chce zariadiť, aby počet platných zmlúv (starých a nových dokopy)
nikdy zo dňa na deň neklesol. Tento cieľ by rád dosiahol použitím čo
najmenšieho počtu nových zmlúv. (Pozor, inšpektorovi nezáleží na dĺžke nových
zmlúv, len na ich počte. Vybavovať novú zmluvu je totiž hrozná otrava).

\nadpis{Formát vstupu}

Prvý riadok vstupu obsahuje číslo $n$ ($1 \leq n \leq 100\,000$). Druhý riadok
obsahuje čísla $a_1, a_2, \dots, a_n$ oddelené medzerami ($1 \leq a_i \leq
10^9$).

\nadpis{Formát výstupu}

Na výstup vypíšte jeden riadok obsahujúci minimálny počet nových zmlúv, ktoré
musí Kloten uzavrieť.


\nadpis{Príklady}

\vstup
4
1 2 2 5
\vystup
0
\koniec

\vstup
4
3 1 2 4
\vystup
2
\koniec

{\sl
Stačia dve nové zmluvy.
Jedno možné riešenie je uzavrieť jednu zmluvu len na druhý deň (teda interval [2,2]) a druhú zmluvu
na druhý a tretí deň (teda interval [2,3]).
Takto dostaneme neklesajúcu postupnosť: 3, 3, 3, 4.
}


\end{document}
