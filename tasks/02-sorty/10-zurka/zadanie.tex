\input ../../../include/include.tex
% proofread by Sameth
% proofread by Jano

\begin{document}

\velkynadpis{Strašná žúrka}

Včera austronautka Maja organizovala strašnú žúrku. Boli tam všetky dôležité zvieratká z lesa. Ráno
sa Maja zobudila s totálnym oknom. A zistila, že niekto jej prehádzal jej pekne uložené trofeje zo
súťaží o stavaní mostov.

Hneď chcela ísť hľadať vinníka, ale nie je si istá, či to nespravila sama, však bohvie, čo včera
robila. Je si však istá, že ak to spravila ona, vymenila najviac dva prvky. Pomôžte jej zistiť, či to
mohla spraviť ona, alebo to bol niekto iný.

\nadpis{Úloha}

Na vstupe dostanete postupnosť celých čísel. Zistite, či sa z nej dá vyrobiť vzostupne 
usporiadaná postupnosť výmenou najviac jednej dvojice prvkov. 

\nadpis{Vstup}

Na prvom riadku je číslo $n$ ($1 \leq n \leq 10^5$) -- počet čísel v postupnosti.

Na druhom riadku je $n$ celých čísel z rozsahu $1$ až $10^9$, určujúce postupnosť.

\nadpis{Výstup}

Ak sa dá postupnosť uporiadať pomocou najviac jednej výmeny, vypíšte \texttt{"Maja"} (bez
úvodzoviek), inak vypíšte \texttt{"neMaja"}.

\nadpis{Príklady}

\vstup
3
5 7 9
\vystup
Maja
\koniec

\vstup
4
2 1 8 12
\vystup
Maja
\koniec

\vstup
4
2 1 12 8
\vystup
neMaja
\koniec

\end{document}
