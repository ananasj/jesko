\input ../../../../include/include.tex

\begin{document}

\velkynadpis{Jabĺčka}

% proofread by Sameth, !! otazka, jezkovych - kratke/dlhe y !! (prezrele -> prezrete, +- ciarky, cisiel -> cisel )
Ježko Pichliačik má véééééľa jabĺčok. Každé z nich má vek, ktorý vie ježko určiť s presnosťou na
nanosekundy. Povedal si, že jabĺčko je zrelé, ak je staré práve $p$ nanosekúnd. Pomôžte ježkovi
rozdeliť jeho jabĺčka na nezrelé, zrelé a prezreté.

\nadpis{Úloha}

Na vstupe máte postupnosť $n$ celých čísiel, udávajúce vek jednotlivých jabĺčok a číslo $p$ --
optimálny vek. Vašou úlohou je rozdeliť túto postupnosť na tri menšie postupnosti. V prvej budú
jabĺčka mladšie ako $p$ nanosekúnd, v druhej tie, ktoré majú presne $p$ nanosekúnd a v tretej tie, ktoré sú
staršie ako $p$ nanosekúnd. Naviac je jedno, v akom poradí vypíšete čísla v rámci jednej
postupnosti.

\nadpis{Vstup}

Na prvom riadku je číslo $n$ a $p$ ($1 \leq n \leq 1\,000\,000$, $1\leq p\leq 10^9$) -- počet
ježkových jabĺčok a optimálny vek jabĺčka.

Na druhom riadku je $n$ kladných celých čísel $v_i$ ($1 \leq v_i \leq 10^9$) určujéce vek ježkových
jabĺčok.

\nadpis{Výstup}

Vypíšte tri riadky. Každý z nich má obsahovať množinu čísel odelených medzerami. Na prvom riadku
majú byť čísla z postupnosti $v_1, v_2 \dots v_n$ menšie ako $p$, na druhom rovné ako $p$ a na
treťom väčšie ako $p$.

\nadpis{Príklady}

\vstup
5 3
1 8 13 5 3
\vystup
1
3
8 13 5
\komentar
Toto samozrejme jediný správny výstup. Posledný riadok môže vyzerať nasledovne: $(5, 8, 13)$, $(5,
13, 8)$, $(8, 5, 13)$, $(8, 13, 5)$, $(13, 5, 8)$ alebo $(13, 8, 5)$. 
\koniec

\vstup
4 4
7 2 8 1
\vystup
2 1

8 7
\koniec

\end{document}
