\input ../../../../include/include.tex

\begin{document}

\velkynadpis{Pichliačov program}

% proofread by Sameth
Ježko Pichliač sa, tak ako vy, učil algoritmy na triedenie. Na konci kapitoly si povedal, že to predsa
nie je ťažké a že vymyslieť vlastný triediaci algoritmus je jednoduché. Hneď aj napísal krátky
pseudokód, ktorý má usporiadať postupnosť čísiel $a_1, a_2, \dots a_n$.

\listing{pichliacov_program.cpp}

Veríte Pichliačovmu programu, alebo viete nájsť protipríklad?

\nadpis{Úloha}

Na vstupe dostanete číslo $n$. Nájdite takú postupnosť $n$ celých čísel z rozsahu $1$ až
$1\,000\,000$, ktorý ježkov program nezoradí správne. Ak taká postupnosť neexistuje, vypíšte $-1$.

\nadpis{Vstup}

Na prvom riadku je jedno celé číslo $n$ ($1 \leq n \leq 100$).

\nadpis{Výstup}

Ak existuje postupnosť $n$ čísel, ktorú ježkov program nezoradí správne, vypíšte ju na jeden riadok.
Inak vypíšte $-1$.

\nadpis{Príklady}

\vstup
1
\vystup
-1
\koniec

\end{document}
