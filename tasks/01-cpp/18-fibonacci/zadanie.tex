\input ../../../include/include.tex

\begin{document}

\velkynadpis{Fibonacci}

Fibonacciho postupnosť je postupnosť čísel, začínajúca číslami 0, 1 a každé ďalšie číslo je súčtom
dvoch predošlých.  Prvé členy Fibonacciho postupnosti teda vyzerajú takto: 0, 1, 1, 2, 3, 5, 8, 13,
21, 34\dots

Na vstupe dostanete jedno číslo $n$ ($1\leq n\leq 30$). Vašou úlohou je vypísať prvých $n$ Fibonacciho čísel.

\textbf{Upozornenie:} Dajte si pozor na formátovanie. Za posledným číslom výstupu už nemá byť
medzera, ale koniec riadka -- \verb!"\n"!.

\nadpis{Príklady}
\vstup
5
\vystup
0 1 1 2 3 
\koniec

\end{document}
