\input ../../../include/include.tex

\begin{document}

\velkynadpis{Egyptské pyramídy}

Tí Egypťania nemali vôbec jednoduchý život. Verte alebo nie, 
v tej dobe ešte neexistovali počítače. A keď nejaký faraón chcel mať pyramídú, 
museli tisíce otrokov vláčiť ťažké kamene.

Vy to máte teraz jednoduché, lebo počítače už existujú.
Takže keď vás nejaký faraón požiada o pyramídu, vy mu ju pekne vyhviezdičkujete na obrazovku.

\nadpis{Úloha}

Na vstupe máte číslo $n$ ($1\leq n\leq 20$), vypíšte $n$ riadkovú pyramídu nasledovne:

Na $i$-tom riadku je $2i-1$ hviezdičiek obkolesených z obochstrán rovnakým počtom medzier tak,
že celý "obrázok" je rovnako široký ako najspodnejší riadok pyramídy.


\nadpis{Príklady}

\vstup
4
\vystup
   *    
  ***  
 ***** 
*******
\komentar
Dávajte si pozor na počet medzier, každý riadok je v tomto
prípade široký 7 znakov, za ktorými nasleduje znak konca riadku \verb'\n'.
\koniec

\end{document}
