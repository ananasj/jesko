\input ../../../include/include.tex

\begin{document}

\velkynadpis{Malá násobilka}

Násobiť z hlavy, to je pre dnešnú mládež veľký problém.
Už nezvládajú ani malú násobilku, takže vás poprosili, aby ste im
ju vypísali na obrazovku.

\nadpis{Úloha}

Vypíšte tabuľku rozmerov $10 \times 10$ s malou násobilkou -- teda v $i$-tom riadku
a $j$-tom stĺpci bude číslo $i$ krát $j$.

Tabuľku pekne naformátujte, teda každé políčko by malo byť široké 4 znaky.
Na takéto formátovanie je dobré použiť funkciu \verb!printf()! s formátovacím 
príkazom \verb!"%<číslo>d"!. Teda napríklad \verb!printf("%5d", 10);! vypíše reťazec \verb!'   10'!.
(tri medzery, znak \verb!'1'! a znak \verb!'0'!)

\nadpis{Príklady}

\vstup
Vstup ignorujete.
\vystup
   1   2   3   4
   2   4   6   8
   3   6   9  12
   4   8  12  16
\komentar
Takto by vyzeral výstup, keby ste mali za úlohu vypísať tabuľku $4\times 4$.
Vy ale máte vypisovať $10\times 10$.
\koniec

\end{document}
