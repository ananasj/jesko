%%
%% This is file `ksp-lst-utf8.sty'.
%%
%% It is a modified version of the file listingsutf8.sty by Heiko Oberdiek <oberdiek@uni-freiburg.de>
%% The file listingsutf8.sty is available at CTAN as a part of the `oberdiek' package bundle.
%%
%% This file (C) 2012 by Michal Forisek <misof@ksp.sk>
%% 
%% This work may be distributed and/or modified under the
%% conditions of the LaTeX Project Public License, either
%% version 1.3 of this license or (at your option) any later
%% version. The latest version of this license is in
%%    http://www.latex-project.org/lppl.txt
%% and version 1.3 or later is part of all distributions of
%% LaTeX version 2005/12/01 or later.
%% 
%% This work has the LPPL maintenance status "author-maintained".
%% 

%% UTF-8 CHARACTERS NOTES:
%%
%% Accented characters ONLY work in listings read from an external file.
%% The inline and display block formats do NOT support accented characters.
%%
%% If you want accented characters, store the program into a separate file and use \listing.
%% The file encoding MUST be UTF-8.
%% All characters used in the file MUST be convertible into the iso-latin2 character set.

%% USAGE EXAMPLES:
%%
%% \listing{filename.cc}          % syntax highlighting guessed for extensions {cc,cpp,pas,py}
%% \listing[Pascal]{filename.ext} % typesets filename.ext using Pascal highlighting
%% \listing[bash]{filename.ext}   % any language supported by listings.sty can be used
%% \listing*[bash]{filename.ext}  % the starred version does not print the "Listing programu" header
%% 
%% In a sentence, use \lstlang{cc}\lstinline!for (int x=0; x<10; ++x) { y+=x; }!.
%%
%% As a display block, use:
%% \lstlang{pas}\begin{lstlisting}
%%   for i:=0 to n-1 do y := y+x;
%% \end{lstlisting}
%%
%% In addition to {cc,cpp,pas,py} any language supported by listings.sty works for \lstlang:
%% \lstlang{Java}\lstinline!import java.util.*;!.

\NeedsTeXFormat{LaTeX2e}
\ProvidesPackage{ksp-lst-utf8}[2012/05/08 KSP listings in (IL2 subset of) UTF-8 v1.0]

%% content copied and tweaked from listingsutf8.sty begin
\begingroup
  \catcode123 1 \catcode125 2
  \def\x{\endgroup
    \expandafter\edef\csname lstU@AtEnd\endcsname{%
      \catcode35 \the\catcode35\relax   \catcode64 \the\catcode64\relax
      \catcode123 \the\catcode123\relax \catcode125 \the\catcode125\relax
    }%
  }%
\x
\catcode35 6 \catcode64 11 \catcode123 1 \catcode125 2
\def\TMP@EnsureCode#1#2{\edef\lstU@AtEnd{\lstU@AtEnd\catcode#1 \the\catcode#1\relax}\catcode#1 #2\relax}
\TMP@EnsureCode{34}{12}  % these are double quotes, changed to catcode 13 by a ksp macro
\TMP@EnsureCode{10}{12}  \TMP@EnsureCode{33}{12}  \TMP@EnsureCode{36}{3}  \TMP@EnsureCode{38}{4}  \TMP@EnsureCode{39}{12}
\TMP@EnsureCode{40}{12}  \TMP@EnsureCode{41}{12}  \TMP@EnsureCode{42}{12} \TMP@EnsureCode{43}{12} \TMP@EnsureCode{44}{12}
\TMP@EnsureCode{45}{12}  \TMP@EnsureCode{46}{12}  \TMP@EnsureCode{47}{12} \TMP@EnsureCode{58}{12} \TMP@EnsureCode{60}{12}
\TMP@EnsureCode{61}{12}  \TMP@EnsureCode{62}{12}  \TMP@EnsureCode{94}{7}  \TMP@EnsureCode{95}{8}  \TMP@EnsureCode{96}{12}
\TMP@EnsureCode{124}{12} \TMP@EnsureCode{126}{13}
\g@addto@macro\lstU@AtEnd{\endinput}
\DeclareOption*{\PassOptionsToPackage\CurrentOption{listings}}
\ProcessOptions*
\RequirePackage{listings}[2002/04/01]
\AtBeginDocument{\@ifundefined{inputencoding}{\RequirePackage{inputenc}}{}}
\RequirePackage{pdftexcmds}[2007/11/11]
\def\lstU@temp#1#2{%
  \begingroup\expandafter\expandafter\expandafter\endgroup
  \expandafter\ifx\csname #1\endcsname\relax
    \PackageWarningNoLine{ksp-lst-utf8}{%
      Package loading is aborted because of missing %
      \@backslashchar#1.\MessageBreak
      #2%
    }%
    \expandafter\lstU@AtEnd
  \fi
}
\lstU@temp{scantokens}{It is provided by e-TeX}
\lstU@temp{pdf@unescapehex}{It is provided by pdfTeX >= 1.30}
\lstU@temp{pdf@filedump}{It is provided by pdfTeX >= 1.30}
\lstU@temp{pdf@filesize}{It is provided by pdfTeX >= 1.30}
\RequirePackage{stringenc}[2007/10/22]
\newif\iflstU@utfviii
\def\lstU@inputenc#1{\expandafter\lstU@@inputenc#1utf8/utf8/\@nil}
\def\lstU@@inputenc#1utf8/#2utf8/#3\@nil{\ifx\\#1\\\lstU@utfviiitrue\def\lst@inputenc{#2}\else\lstU@utfviiifalse\fi}
\lst@Key{inputencoding}\relax{\def\lst@inputenc{#1}\lstU@inputenc{#1}}
\def\lstU@input#1{%
  \iflstU@utfviii
    \edef\lstU@text{\pdf@unescapehex{\pdf@filedump{0}{\pdf@filesize{#1}}{#1}}}%
    \StringEncodingConvert\lstU@text\lstU@text{utf8}\lst@inputenc
    \def\lstU@temp{\scantokens\expandafter{\lstU@text}}%
  \else
    \def\lstU@temp{\input{#1}}%
  \fi
  \lstU@temp
}
\def\lstU@temp#1\def\lst@next#2#3\@nil{\def\lst@InputListing##1{#1\def\lst@next{\lstU@input{##1}}#3}}
\expandafter\lstU@temp\lst@InputListing{#1}\@nil
%% listingsutf8.sty end

%% By default, listings.sty gets the fonts wrong for some characters.
%% This looks ugly once we start using \ttfamily fonts.
%% Below are ugly low-level fixes for the characters - < > _ { }

\lst@CCPutMacro\lst@ProcessOther{"2D}{\lst@ttfamily{-{}}{-{}}}\@empty\z@\@empty
\lst@CCPutMacro\lst@ProcessOther{"3C}{\lst@ttfamily{<{}}{<{}}}\@empty\z@\@empty
\lst@CCPutMacro\lst@ProcessOther{"3E}{\lst@ttfamily{>{}}{>{}}}\@empty\z@\@empty
\lst@CCPutMacro\lst@ProcessOther{"5F}{\lst@ttfamily{\char '137{}}{\char '137{}}}\@empty\z@\@empty
\lst@CCPutMacro\lst@ProcessOther{"7B}{\lst@ttfamily{\char '173{}}{\char '173{}}}\@empty\z@\@empty
\lst@CCPutMacro\lst@ProcessOther{"7D}{\lst@ttfamily{\char '175{}}{\char '175{}}}\@empty\z@\@empty

%% Below we have our general settings for listings in all languages.

\lstset{
    basicstyle=\fontfamily{pcr}\fontseries{m}\fontshape{n}\fontsize{7}{9}\selectfont, % a very small \tt font
    keywordstyle=\fontseries{b}\fontshape{n}\selectfont, % the same but in bold
    commentstyle=\fontseries{m}\fontshape{sl}\selectfont, % the same but slanted
    tabsize=8,
    showtabs=false,
    inputencoding=utf8/latin2,
    lineskip=-2pt, % more dense baselines
    keepspaces=true, % for instance to keep "/* comment" from turning to "/*comment"
}

%% Finally, we define the commands for users of this package.

\def\@lstheader#1{\par\ifdim\lastskip>\parskip\else\vskip0.4cm\relax\fi\noindent{\sf\bfseries Listing programu\ifnum\pdfstrcmp{#1}{}=0\relax\else\ (#1)\fi}\par\smallskip}

\def\lstlang#1{%
    \lstset{language={}}%
    \ifnum\pdfstrcmp{#1}{cc}=0\lstset{language=C++}\else
    \ifnum\pdfstrcmp{#1}{cpp}=0\lstset{language=C++}\else
    \ifnum\pdfstrcmp{#1}{pas}=0\lstset{language=Pascal}\else
    \ifnum\pdfstrcmp{#1}{py}=0\lstset{language=Python,morekeywords={as,assert,with,yield}}\else
    \lstset{language=#1}\fi\fi\fi\fi
}

\newif\iflisting@star
\newcommand{\listing}{\@ifstar{\listing@startrue\listing@i*}{\listing@starfalse\listing@i*}}
\def\listing@i*{\@ifnextchar[{\listing@ii*}{\listing@ii*[@]}}
\def\listing@ii*[#1]#2{%
    \def\hllang{}%
    \ifnum\pdfstrcmp{#1}{@}=0% no optional argument provided, try to guess language from extension
        \ifnum\pdfmatch{[.]cc$}{#2}=1\def\hllang{C++}\lstset{language=C++}\fi
        \ifnum\pdfmatch{[.]cpp$}{#2}=1\def\hllang{C++}\lstset{language=C++}\fi
        \ifnum\pdfmatch{[.]pas$}{#2}=1\def\hllang{Pascal}\lstset{language=Pascal}\fi
        \ifnum\pdfmatch{[.]py$}{#2}=1\def\hllang{Python}\lstset{language=Python,morekeywords={as,assert,with,yield}}\fi
        \ifnum\pdfstrcmp{\hllang}{}=0\def\hllang{Text}\lstset{language={}}\fi
    \else
        \def\hllang{#1}\lstset{language=#1}%
    \fi
    \iflisting@star\relax\else\@lstheader{\hllang}\fi% make a header if not in the starred version
    \lstinputlisting{#2}%
}

\lstU@AtEnd
\endinput
%%
%% End of file `ksp-lst-utf8.sty'.
