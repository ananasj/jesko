%Tu si môžete zaznačiť, že pracujete na danej veci. V prípade, že ste napísali len časť a ďalej už
%nechcete, alebo ste hotoví tak sa odtiaľ odpíšte. Bolo by však fajn, aby jedu vec robil jeden
%človek ak celok a zvyšní len kontrolovali
%vypracuva: Askar
\input ../../include/include.tex

\begin{document}

\velkynadpis{Dátové štruktúry nad poľom}
\textit{Návod na použitie: očakáva sa, že čítateľ už niečo vie o pojmoch 
''časová(pamäťová) zložitiosť'', ''pole''. Ak takéto
pojmy nepoznáš, tak Ti odporúčam najprv preštudovať príslušné časti tejto 
úžasnej kuchárky.}


Obsah:
\begin{itemize}
    \item načo nám sú dátové štruktúry
    \item stack    
    \item queue
    \item deque
    \item linked-list
\end{itemize}

\medskip

Dátovú štruktúru si môžeme neformálne predstaviť ako modrú krabičku, do ktorej 
vkladáme nejake údaje (napríklad čísla, reťazce, súradnice kozmických lodí v 
$5$-dimenziálnom priestore, atď), a potom pomocou vstavaných tlačitiek sa 
môžeme opýtať na nejakú informáciu o nich. Môže to byť čokoľvek (tu a ďalej: 
bavíme sa o krabičkách, ktoré ako vstup berú čísla): počet dvojok spomedzi 
prídaných čisel, počet párnych čísel, najmenšie číslo, najväčšie číslo, prvé 
alebo posledné prídané číslo, priemer nepárnych čísel, atď.
Druhou možnosťou často je schopnosť nejakým spôsobom (pomocou iných tlačitiek) 
meniť tieto údaje. Napríklad: prídať ďalší prvok, zvýšiť všetky čísla o $15$, 
vyhodiť všetky päťky, prepísať druhé najmenšie číslo na iné, atď. 

Používateľov takýchto modrých krabičiek väčšinou nezaujíma, čo sa nachádza vo 
vnutri. Stáči, aby to fungovalo spoľahlivo a rýchlo. No a tu prichádza odpoveď na 
otázku ''Načo potrebujeme dátové štruktúry?''. Nie každý spôsob ukladania dát 
môže dovoliť rýchlo odpovedať na takéto requesty. Potrebujeme špecifikovať spôsob 
spracovania dát v závislosti od požiadaviek na danú modrú krabičku, aby každé 
''tlačitko'' pracovalo rýchlo (tu a ďalej: s optimálnou časovou a pamäťovou 
zložitosťou vzhľadom na množstvo prvkov v danej DŠ).

\nadpis{Zásobník}

Záčneme DŠ (dátovou štruktúrou (modrou krabičkou)), ktorá má následovné tlačitká 
(funkcie): 
$PUSH(x)$, $TOP()$, $POP()$, $SIZE()$. \\
$PUSH(x)$ pridáva prvok $x$ do DŠ, $TOP()$ zobrazuje posledný podľa času 
prídania prvok v DŠ, $POP()$ vyhadzuje z DŠ posledný podľa času prídania prvok 
(ten, čo ukazuje funkcia $TOP()$), $SIZE()$ ukazuje aktuálny počet prvkov v DŠ.

Znie to veľmi múdro, ale v skutočnosti chceme modrú krabičku, ktorá by simulovala 
proces pridávania (a odstraňovania) prvkov do krabice chipsov Pringles (\textit{''Once 
you pop, you can't stop!''} ) s podmienkou, že ''šírka'' prvkov je takmer
rovna priemeru tejto trubici, a teda žiadne dva prvky si nevedia vymeniť poradie, 
kým sú vo vnútri trubice. Dá sa toto tiež predstavovať ako proces pridávania 
nábojov do zásobníku zbrane, odkiaľ vlastne aj plynie názov, ale analógia s Pringles
je presnejšia a vystižnejšia, lebo pri každej operácii vkládania do zásobníku zbrane sa 
hybú všetky náboje, čo ale vôbec nie je optimálne a v žiadnom prípade to simulovať nebudeme. 

Naším cieľom je teda naprogramovať takúto krabičku s podmienkou, že každá z 
týchto 4 operácií je rýchla. 

\medskip

Najľahšia je implementácia pomocou obyčajného poľa. Najprv odhadneme maximálnu veľkosť 
zásobníka. Nech bude rovná $N$. Prvky budeme postupne ukladať na pozíciách $0$ až $N-1$. 
A teda, čím neskôr je prvok vložený, tým väčšie je číslo políčka, kde bude uložený.
Spolu s poľom budeme v premennej $size$ pamätať aktuálne množstvo prvkov v zásobníku.
Teda, ak máme $size = 5$, tak máme prvky uložené v políčkach $0$ až $4$.
   
Funkcia $SIZE()$ sa potom implementuje jednoducho vypísaním premennej $size$.

Funkcia $TOP()$ sa realizuje tak, že vypíšeme číslo na políčku $size-1$, 
lebo tam je uložený posledný podľa času pridania prvok.

Funkcia $PUSH(x)$ sa realizuje tak, že zapíšeme $x$ na políčko $size$ a zväčšíme premennú $size$ 
o $1$ (všetky prvky, čo boli pridané pred tým, sú uložené na pozíciách $0$ až 
$size-1$, čiže pridanie nového prvku na políčko $size$ je úplne spravné, lebo 
to zachová náš invariant). 

Funkcia $POP()$ sa realizuje tak, že jednoducho zmenšíme 
premennú $size$ o jedna (môžeme síce ten prvok aj premázať, ale načo? Keď sa 
tam dostaneme nabudúce, tak tam zapíšeme už nový prvok).

Toto je jednoduchá implementácia vyše opísaného:
\listing{stack/stack_easy.cc}

Toto je to isté, len celý zásobník je spravený ako poriadna štruktúra:
\listing{stack/stack_easy_nice.cc}

\end{document}
