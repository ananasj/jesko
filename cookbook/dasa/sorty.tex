\input ../../include/include.tex

\begin{document}

\velkynadpis{Triedenie}

Obsah:
\begin{itemize}
    \item cieľ
    \item $n^2$ sorty
    \item $n\log n$ sorty
    \item nejde to lepšie
    \item $n$ sorty
    \item STL
\end{itemize}

\medskip

Ako napovedá názov textu, budeme sa rozprávať o triedení. Schopnosť zoradiť veci do poradia je totiž
veľmi dôležitá a využívaná. Však stačí, že chcete zoradiť futbalové družstvá podľa počtu bodov čo
získali. Alebo hráte karty a chcete ich mať na ruke pekne poporadí od najmenšej po najväčšiu. Pri
tomto všetkom potrebujeme usporiadať veci podľa určitej vlastnosti a od nás informatikov sa čaká, že
to budeme vedieť robiť rýchlo a efektívne a to aj pre viac ako $14$ kariet na ruke, alebo $30$ tímov
vo futbalovom rebríčku.

\medskip

Na oboznámenie sa s utriedenými postupnosťami, skúste vyriešiť úlohu \textbf{poriadok}.

\nadpis{Pomalšie triedenia}

Začnime s ľahšími (na nakódenie aj pochopenie), ale pomalšími algoritmami, riešiacimi problém
triedenia. Ten môžeme chápať nasledovne: V poli $A$ máme $n$ prvkov. Chceme preusporiadať pole $A$
tak, aby v ňom prvky boli od najmenšieho po najväčší.

Dôležité je si určiť, čo znamená najmenší prvok a vôbec operátor \texttt{menší}. Keď porovnávame
čísla, je to jasné. Používame normálne $<$, teda $2<8$, $47<212$ \dots Nie vždy však pracujeme len s
číslami. Môžeme dostať napríklad písmená. Tie môžeme porovnávať podľa poradia v abecede. Samozrejme
to nie je všetko, čo môžeme robiť. Našu operáciu \texttt{menší} si môžeme zvoliť podľa ľubovôle,
stačí aby nám dávala zmysel a po celý čas sme sa jej držali. V ďalšom texte však budeme
predpokladať, že pracujeme s celými číslami a používame normálne $<$.

\medskip

Aby som vás týmto úplne neunudil, vyriešte úlohu \textbf{vystava}.

\medskip

Prečo práve takúto úlohu? Dostať najmenší prvok na začiatok poľa a nič iné nemeniť? Vedie nás to
totiž k našému prvému triediacemu algoritmu -- \texttt{Minsort}. Jeho myšlienka je veľmi jednoduchá.
Zoberme si pole $A$ celých čísiel a chceme ho usporiadať od najmenšieho po najväčší prvok. Ktoré
číslo sa bude nachádzať na začiatku takto utriedenej postupnosti? Samozrejme, že to najmenšie. Ak by
to tak nebolo, potom by niektorá časť poľa predsa nebola utriedená. Vieme teda, čo je na prvej
pozícii výsledného poľa. Zvyšných $n-1$ čísiel, ktoré sú teraz na pozíciách $2$ až $n$ je však stále
neusporiadaných.

Tu sa však dá použiť znova tá istá myšlienka. Najmenšie číslo, ktoré je teraz na pozícii $1$ necháme
tak, to už je umiestnené dobre. Skúsime nájsť číslo na pozíciu $2$. A to bude samozrejme najmenšie
číslo z čísiel zvyšných, teda tých, čo sú na pozíciách $2$ až $n$. A takto postupujeme ďalej a
ďalej. Postupne priraďujeme čísla na prvú, druhú, tretiu až $n$-tú pozíciu a vždy ho nájdeme ako
minimum z čísiel zvyšných. Znamená to, že $n$ krát urobíte to, čo ste robili v úlohe
\textbf{vystava}, akurát na stále menšom poli. A to už predsa nie je problém.
Výsledná časová zložiťosť však bude $O(n^2)$. Nie zlé, ale stále nie najlepšie.

\medskip

Predpokladám, že teraz už bez problémov viete naprogramovať vlastný \texttt{Minsort}, môžete si to
teda hneď vyskúšať na úlohe \textbf{utried1}.

\medskip

Asi vás veľmi neprekvapivý, že na úlohu sa dá pozrieť aj opačne. Aké číslo je na poslednom mieste
zoradeného poľa. Samozrejme, že je to maximum, čo dá vznik algoritmu \texttt{Maxsort}.

\bigskip

Zostaňme ešte chvíľu pri triedeniach so zložitosťou $O(n^2)$. Na začiatku sme spomínali ukladanie
kariet. Tento proces je všetkým známy. Postupne, jednu po jedne, beriete karty z vašej kôpky a
nachádzate im príslušné miesto na ruke. Ako však nájdete miesto, kam daná karta patrí? Začnete na
konci. Porovnáte novú kartu s poslednou kartou v ruke. Ak je nová karta väčšia, rovno ju dáte na
koniec, lebo viete, že všetky ostatné karty na ruke sú od nej tiež menšie. Ak je však menšia ako
posledná karta, znamená to, že ju treba zaradiť niekam ďalej. Pozriete sa teda na predposlednú
kartu. Takto postupujete, až kým prvýkrát nie je nová karta väčšia ako karta, na ktorú sa pozeráte,
alebo ste dosiahli úplný začiatok. Na toto miesto kartu vsuniete.

Tento istý prístup sa dá simulovať aj programom. Postupne beriete čísla a máte pole, kde si ukladáte
výsledok. Postupne idete od konca poľa a sledujete, či vaše nové číslo nie je väčšie. Takto mu
nájdete správne miesto. Problém však je, že neviete toto číslo strčiť do stredu poľa tak ako kartu.
Toto môžete vyriešiť tak, že vždy keď zistíte, že nové číslo je menšie ako číslo skúmané, posuniete
toto číslo o jedna dozadu v poli. Takto si spravíte priestor pre nové číslo, ktoré tam potom viete
pohodlne vložiť. Tento algoritmus má tiež zložitosť $O(n^2)$ a volá sa \texttt{Insertsort}. Kvôli 
problematickému vsúvaniu čísiel sa nedá viac vylepšiť. Jeho naprogramovanie si tiež môžete vyskúšať
na úlohe \textbf{utried1}, dajte si však pozor, aby ste skontrolovali, či nie ste už na začiatku poľa.

Toto samozrejme nie je posledný zo série triediacich algoritmov so zložitsťou $O(n^2)$. Existuje ich
ešte o dosť viac, od šialených až po tie rozumnejšie. Nie sú však až tak dôležité, vrhnime sa preto
na algoritmy rýchlejšie.

\nadpis{Rýchle triedenia}

\end{document}
