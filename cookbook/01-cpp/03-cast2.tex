%Tu si môžete zaznačiť, že pracujete na danej veci. V prípade, že ste napísali
%len časť a ďalej už nechcete, alebo ste hotoví tak sa odtiaľ odpíšte. Bolo by
%však fajn, aby jedu vec robil jeden človek ak celok a zvyšný len kontrolovali
%vypracuva: Jano
%gram. opravy Maja
\input ../../include/include.tex

\begin{document}

\velkynadpis{Ďalšie časti C++}

\textit{Ako správne používať tento študijný text: Milý čitateľ, chystáš sa
prečítať si kuchárku o základoch programovacieho jazyka C++. Najdôležitejšie
pri učení sa nového programovacieho jazyka je poriadne si ho precvičiť. Preto
počas čítania narazíš na niekoľko úloh, ktoré ti odporúčame vyriešiť a
naprogramovať.  Má to dopomôcť k tomu, aby si si všetko lepšie zapamätal a dostal
do krvi.}

\textit{Tento text je napísaný aj pre ľudí, ktorí nikdy program nevideli. Pokiaľ
už máš nejaké základy z programovania, môžeš pokojne preskakovať tie časti,
ktoré ovládaš.}

\nadpis{Obsah:}
\begin{itemize}
    \vspace{-8pt}
    \item funkcie
    \vspace{-5pt}
    \item znaky
    \vspace{-5pt}
    \item reálne čísla
    \vspace{-5pt}
    \item matematika
\end{itemize}

\medskip


\kapitola{Funkcie}
\textit{Ako neprogramovať jednu vec viackráť}

Doteraz, keď sme v programovaní chceli niečo zopakovať viackrát, tak
sme použili cyklus. Niekedy však potrebujeme danú vec robiť na rôznych miestach.
Napríklad vypisovať čísla 1 až 10 v rôznych častiach programu.

\lstlang{cpp}\begin{lstlisting}
int main(){
    // radi by sme napisali desat cisel na zaciatku programu
    for(int i = 0; i<10; ++i)
        printf("%d ", i+1);
    printf("\n");

    int n;
    scanf("%d", &n);
    if (n < 0) {
        // potom raz v pripade, ze uzivatel zada zaporne cislo
        for(int i = 0; i<10; ++i)
            printf("%d ", i+1);
        printf("\n");
    }

    for(int j = 0; j<n; ++j) {
        // alebo n krat, ked uzivatel zada kladne n
        for(int i = 0; i<10; ++i)
            printf("%d ", i+1);
        printf("\n");        
    }
}
\end{lstlisting}

Čo je na vyššie uvedenom programe zle?

Veľakrát používa rovnaký kód. A to máme ešte štastie, že opakujeme len trikrát
a opakujúci sa kód je dlhý len tri riadky. Aj tak, keď si zmyslíme, že nechceme
vypisovať 10 čísel ale 11, tak to treba všade prepísať. Fuj.

Teraz si predstavme, že robíme niečo podstatne zložitejšie a na 200 miestach by
sme mali rovnaký 10 riadkový blok. Keby sme tam chceli niečo zmeniť, tak sa
zbláznime.  A určite pri nejakom prepise spravíme chybu. A nemáme šancu tú
chybu nájsť.

Proste copy-pastovanie kódu je zlo a chceme sa mu vyhnúť, lebo tvorí
neprehľadné a chybné programy. Ale ako sa mu vyhneme?

\medskip

Celkom dobrý spôsob je zabaliť kus kódu do funkcie -- nejako si ten kus kódu
nazvať.  Funkciu môžeme chápať ako samostatný program. Proste postupnosť
príkazov, ktorá niečo robí.  Konieckoncov všetky naše doterajšie programy boli
len funkcia \verb!int main()!.

Funkciu vyrobíme tým, že ju zadeklarujeme v tvare \verb!<typ> <meno>(<argumenty>){<príkazy>}!
Spúšťame (voláme, po anglicky call) funkcie \verb!<meno>(<hodnoty-argumentov>)!

Podivný program zhora vieme prepísať napríklad takto:

\lstlang{cpp}\begin{lstlisting}
// nasledujuca funkcia je typu void, vola sa vypis_desat_cisel a nema ziadne argumenty

void vypis_desat_cisel(){
    for(int i = 0; i<10; ++i)
        printf("%d ", i+1);
    printf("\n"); 
}

// s funkciou main uz sme sa stretli
int main(){
    // zavolame funkciu
    vypis_desat_cisel()
    
    int n;
    scanf("%d", &n);
    if (n < 0)
        vypis_desat_cisel()

    for(int j = 0; j<n; ++j) 
        vypis_desat_cisel()
}
\end{lstlisting}

Dostali sme oveľa prehľadnejší kód a ak chceme napríklad zmeniť počet
vypisovaných čísel, stačí to spraviť na jednom mieste.

\medskip 

Čo je to \verb!void! a vôbec ten typ?

Funkcia môže vracať nejakú hodnotu a \textit{ten} typ je typ hodnoty, akú
funkcia vracia.  \verb!void! znamená, že funkcia nevracia nič. Funkcia, ktorá
niečo vracia, môže byť napríklad funkcia \verb!osem()! z nasledujúceho
príkladu. Isto si všimnete, že na vrátenie hodnoty slúži príkaz \verb!return!.
Príkaz \verb!return! zároveň skončí vykonávanie funkcie, preto všetko, čo je za
ním, sa už nevykoná. 

\lstlang{cpp}\begin{lstlisting}
int osem(){
    return 8;
}

int devat(){
    return 9;
    printf("tento prikaz sa nevykona\n");
}

void nic_nerob(){
    return; // pri void funkcii nemame co vratit, tak piseme return; bez hodnoty
    printf("ani tento\n");
}

int main(){
    // navratove hotnoty mozeme vyuzivat vo vyrazoch:

    printf("funkcia osem vracia %d\n", osem());
    int a = devat();
    printf("a je %d\n", a);

    // V jazyku C sa musel na konci main() pisat aj tento riadok:
    return 0;
    // V C++ pokial nic nevratime, vrati sa nejaka defaultna hodnota, napr. 0,
    // takze ten riadok pisat nemusime
}   
\end{lstlisting}

\medskip
    
Zatiaľ tie funkcie nie sú veľmi úžasné, lebo robia stále to isté (asi ako
programy bez vstupu).  A stále nevieme, čo sú to tie argumenty.

Tak poďme obe veci vyriešiť zároveň. Argumenty sú akoby vstup pre funkciu.  Už
sme sa s tým stretli, keď sme napríklad funkcii \verb!printf()! do zátvoriek
zadávali, čo má vypísať.  Analogicky namiesto funkcie
\verb!vypis_desat_cisel()! môžeme použiť všeobecnejšiu funkciu:

\lstlang{cpp}\begin{lstlisting}
void vypis_cisla(int pocet){
    for(int i = 0; i<pocet; ++i)
        printf("%d ", i+1);
    printf("\n"); 
}

int main(){
    vypis_cisla(10);

    // takto vieme pouzivat funkciu na rozne pocty vypisovanych cisel
    vypis_cisla(10 + 7);
    int n;
    scanf("%d", n);
    vypis_cisla(n);
}
\end{lstlisting}

Funkcia môže mať ľubovoľný počet argumetov, pri deklarácii(vyrábaní) funkcie
ich jednoducho vymenujeme medzi zátvorky \verb!()! v rovnakom tvare, ako pri
deklarácii premenných, oddelené čiarkami. Následne ich môžeme používať ďalej vo
funkcii, ako nové premenné (t.j. ich môžeme napríklad aj meniť).

Pri volaní funkcie píšeme medzi zátvorky hodnoty alebo výrazy, samozrejme musí
sa zachovať typ.

\lstlang{cpp}\begin{lstlisting}
int scitaj(int a, int b, int c){
    // prehodíme hodnoty premennych, len tak pre srandu
    int pomocna = a;
    a = b;
    b = c;
    c = pomocna;
    return a+b+c;
}
int main(){
    int a = 5;
    printf("%d\n", scitaj(1+7, 4*a, a)
}
\end{lstlisting}

\medskip

Veľmi užitočná vlasnosť funkcií je rekurzívne volanie, teda, že funkcia sa može
volať opakovne sama seba. Toto vieme využiť napríklad aj na počítanie
všeliakých zaujímavých vecí bez toho aby sme museli nejak významne
premýšľať.

Fibonacciho postupnosť, $\{F_n\}$,  je veľmi zaujímavá číselná postupnosť,
definovaná takto: $F_0 = 0$, $F_1 = 1$, ak $i>1$ potom$F_i = F_{i-1} +
F_{i-2}$.  Bez toho aby sme tejto postupnosti hlbšie porozumeli, môžeme ju
zapísať ako funkciu.

\lstlang{cpp}\begin{lstlisting}
int F(int n){
    if (n < 2)
        return n;
    return F(n-1) + F(n-2);
}
\end{lstlisting}

Zjavne \verb!F(n)!$ = F_n$. Nie je to super? Jediný problém tejto funkcie je,
že nie je veľmi rýchla, na vašich počítačoch sa rozumne rýchlo zvládajú čísla
pre $n \leq 40$. Neskôr sa naučíme, ako sa tomuto neduhu pri funkciách vyhnúť.

\medskip

Funkcia sa môžu volať aj navzájom, \textbf{nie} však takto:
\lstlang{cpp}\begin{lstlisting}
void a(){
    b(); // ! chyba ! funkcia b nie je deklarovana.
}
void b(){
    a();
}
\end{lstlisting}

Volať funkciu môžeme až po deklarácií, ale naštastie môžeme deklarovať
funkciu pred definovaním (napísaním tela) takto:

\lstlang{cpp}\begin{lstlisting}
void b();

void a(){
    b();
}
void b(){
    a();
}
\end{lstlisting}

Samozrejme tieto funkcie \verb!a!, \verb!b! nemajú veľký úžitok, lebo po
zavolaní jednej z funkcií sa program zacyklí až sa časom minie pamäť do ktorej
sa zapísujú zavolané funkcie a program spadne.

\cvicenie Napíšte funkciu, ktorá počíta $n! = 1\cdot 2\cdot\dots\cdot
(n-1)\cdot n$ bez použitia cyklov.

\riesenie 
\lstlang{cpp}\begin{lstlisting}
int faktorial(int n){
    if (n < 2) return 1;
    return n*faktorial(n-1);
}
\end{lstlisting}

\kapitola{Znaky}

V C++ okrem čísel vieme pracovať aj so znakmi. Znaky píšeme medzi jednoité
úvodzovky, napríklad \verb!'a'! \verb!'z'! \verb!'F'! \verb!'7'! \verb!'.'!
\verb!'#'! \verb!'-'!\dots Niektoré znaky treba písať pomocou \verb'\', ako sme
si už spomínali, napríklad \verb!'\n'! alebo \verb!'\''!.

Čo sa týka premenných, najlepšie je používať typ char, ktorý je v skutočnosti
číselná premenná s rozsahom -128 až 127. Každé z týchto čísel má priradený
nejaký znak podľa ASCII \url{http://sk.wikipedia.org/wiki/ASCII}. 

Napríklad 64 je \verb!'A'!, ale takéto veci si nemusíme pamätať,
lebo so znakmi sa dajú robiť matematické operácie ako s číslami. 
Napríklad \verb!'a'+1 == 'b'!.

Ak chceme znak vypísať ako znak, tak použijeme značku \verb!%c!. 

\cvicenie Skúste si spustiť nasledovné programy a porovnajte ich správanie:

\lstlang{cpp}\begin{lstlisting}
#include<cstdio>

int main(){
    char a;
    scanf("%c", &a);
    printf("<%c>\n", a);
}
\end{lstlisting}
\lstlang{cpp}\begin{lstlisting}
#include<iostream>
using namespace std;

int main(){
    char a;
    cin >> a;
    cout << "<" << a << ">" << endl;
}
\end{lstlisting}

\riesenie Prvý program načítava všetky riadky, teda aj biele znaky
(medzery, tabulátory, konce riadkov). Druhý program biele znaky preskakuje.
V prípade, že by sme v prvom programe napísali miesto \verb!scanf("%c", &a);!
\verb!scanf(" %c", &a);! mal by sa správať rovnako ako druhý.

\medskip

Pole znakov, teda \verb!char meno[]!, môžeme nazývať reťazec. 
Napríklad \verb!"jesko"! je reťazec a zároveň je to pole
\verb!{'j', 'e', 's', 'k', 'o', '\0'}!. \verb!'\0'! je znak s hodnotou 0
a v C++ sa ním ukončujú všetky reťazce. 

Všetky funkcie, ktoré robia niečo s reťazcami, končia čítanie reťazca
na nulových znakoch. Preto aj príkaz \verb!printf("a\0hoj");! vypíše len \verb!"a"!. 

Jeden zo spôsobov ako vyrábať nami definované reťazce je napr.
\verb!char retazec[] = "lubovolny text";!

\cvicenie Ako jedným príkazom skrátime reťazec 
\verb!char str[] = "dlhy retazec";! na dĺžku 3?

\riesenie \verb!str[3] = 0;! alebo \verb!str[3] = '\0';!.

\medskip

Načítavať a vypisovať reťazce vieme pomocou značky \verb!%s! s tým rozdielom,
že pri načítavaní nepíšeme \verb!&!.

\lstlang{cpp}\begin{lstlisting}
#include<cstdio>

int main(){
    char a[100];
    scanf("%s", a);
    printf("<%s>\n", a);
}
\end{lstlisting}
\lstlang{cpp}\begin{lstlisting}
#include<iostream>
using namespace std;

int main(){
    char a[100];
    cin >> a;
    cout << "<" << a << ">" << endl;
}
\end{lstlisting}

Ako si iste všimneme po spustení programov, reťazce sa načítavajú "po slovách",
vždy len po najbližší biely znak (medzera, koniec riadku, tabulátor).

\medskip

Pri reťazcoch (ale aj iných poliach) a operátoroch si treba dávať pozor.  Keď
\verb!A! a \verb!B! sú polia, tak \verb!A<B! \textbf{ne}vráti \verb!true! v
prípade, že A obsahuje menši čísla. Rovnako aj \verb!A == B! \textbf{ne}testuje
či polia obsahujú rovnaké čísla. Ani \verb!A + B! \textbf{ne}sčíta obsahy polí
ani \verb!A = B!  \textbf{ne}prekopíruje čísla z jedného do druhého. Jednoducho
operátory robia s poľami celkom odlišné veci.

Je to spôsobené tým, že hodnota premenných \verb!A! a \verb!B! sú smerníky do
pamäte a operátory pracujú len s týmito smerníkmi.

\kapitola{Reálne čísla}

Okrem celých čísel máme v C++ aj reálne. Pre ne máme typ
\verb!double!, načítavame a vypisujeme ich pomocou značky \verb"%lf",
pri výpise vieme ovplyvniť počet vypísaných desatinným miest \verb"%<pocet>lf"
ako vidíme v príklade:

Keď chceme napísať nejakú reálnočíselnú hodnotu, môžeme buď pretypovať
\verb!double(7)! alebo napísať desatinnú bodku \verb!7.0!.

\lstlang{cpp}\begin{lstlisting}
int main(){
    double a;
    scanf("%lf", &a);
    printf("%2lf\n", a);
    cin >> a;
    cout << a << endl;
}
\end{lstlisting}

\medskip

Pri práci s reálnymi číslami si musíme dávať obrovský pozor na presnosť.
Napríklad $(5.1 - 3.1 - 2.0 == 0.0)$ vráti $false$. Počas výpočtov totiž
dochádza k drobným zaokúhľovaniam.

Ilustrovať si to vieme nasledovným príkladom: majme kalkulačku, ktorá počíta v
desiatkovej sústave a má presnosť 3 desatinné miesta.  V matematike $1 - 1/3 -
1/3 - 1/3 = 0$, ale v kalkulačke $1.000 - 0.333 - 0.333 - 0.333 = 0.001 \neq
0.000$. 

V skutočnosti však double nemá presnosť na nejaký počet desatinných miest, ale
na nejaký počet cifier. Keby naša kalkulačka mala presnosť na 4 cifry, tak
543210 by sa zaokrúhlilo na 543200. Takže keď používame doubley, tak nastávajú
nepresnosti aj v celých číslach.

Podobne napríklad vďaka tomu vieme nájst nenulové $a,b$ také, že $a + b = a$.
Napríklad $1000 + 0.01 = 1000$. Takisto výsledok $a + b + c$ môže závisieť od
poradia sčiťovania. $(1000 + 0.4) + 0.4 = 1000 + 0.4 = 1000$, ale $1000 + (0.4
+ 0.4) = 1000 + 0.8 = 1001$.

\medskip

\verb!double! ukladá 52 binárnych cifier, teda približne 15 desiatkových.  Pri
väčšom počte operácií presnosť výsledku klesá.

Najväčšie číslo, aké je \verb!double! schopný reprezentovať je približne
$10^{300}$.

\medskip

Dôležité je nepoužívať \verb!a == b! ale miesto toho napríklad
\verb!abs(a-b)<1e-8!.  \verb!1e-8! je zhruba $0.1^8 = 0.00000001$. \verb!abs!
je funkcia, ktorá vracia absolútnu hodnotu (odstráni znamienko) a nachádza sa v
knižnici \verb!cmath!.

\kapitola{Matematika}

V knižnici \verb!cmath! sa nachádza aj zopár ďalších zaujímavých funkcií. 
Napríklad \verb!sqrt(x)! $= \sqrt{x}$, \verb!exp(x)! $= e^x$, \verb!log(x)! je prirodzený
logaritmus $x$, pozná aj goniometrické funkcie \verb!sin(x)!, \verb!cos(x)! a mnoho ďalších,
ktoré nájdete na \url{http://www.cplusplus.com/reference/cmath/}. 
Všetky pracujú s reálnymi číslami.

\lstlang{cpp}\begin{lstlisting}
#inclide<cstdio>
#include<cmath>

int main(){
    double d;
    scanf("%lf", &d);
    printf("%lf\n", sqrt(d));
}
\end{lstlisting}

%\lstlang{cpp}\begin{lstlisting}
%\end{lstlisting}






\end{document}


