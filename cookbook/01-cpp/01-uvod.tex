%Tu si môžete zaznačiť, že pracujete na danej veci. V prípade, že ste napísali
%len časť a ďalej už nechcete, alebo ste hotoví tak sa odtiaľ odpíšte. Bolo by
%však fajn, aby jedu vec robil jeden človek ak celok a zvyšný len kontrolovali
%vypracuva: Jano
%gram. proofread Maja  
\input ../../include/include.tex

\begin{document}

\velkynadpis{Úvod do C++}

\textit{Ako správne používať tento študijný text: Milý čitateľ, chystáš sa
prečítať si kuchárku o základoch programovacieho jazyka C++. Najdôležitejšie
pri učení sa nového programovacieho jazyka je poriadne si ho precvičiť. Preto
počas čítania narazíš na niekoľko úloh, ktoré ti odporúčame vyriešiť a
naprogramovať.  Má to dopomôcť k tomu, aby si si všetko lepšie zapamätal a dostal
do krvi.}

\textit{Tento text je napísaný aj pre ľudí, ktorí nikdy program nevideli. Pokiaľ
už máš nejaké základy z programovania, môžeš pokojne preskakovať tie časti,
ktoré ovládaš.}

Obsah:
\begin{itemize}
    \item čo je to program?
    \item kompilácia - Linux
    \item kompilácia - Windows %TODO
    \item náš prvý program
    \item premenné
\end{itemize}

\medskip

\kapitola{Program} \textit{O čom tu celý čas rozprávame? Čo je to program? Čo
je to algoritmus? Čo je to programovací jazyk?  Ak si myslíš, že tieto základy
už vieš, môžeš túto sekciu preskočiť, inak si ju radšej prečítaj.}

\textbf{Algoritmus} je postupnosť niekoľkých dobre definovaných inštrukcií --
úkonov, ktorá slúži na vykonanie nejakej úlohy.  Napríklad algoritmus na
uvarenie čaju môže byť nasledovný:

\texttt{Zoberieme pohár, vrecúško čaju a rýchlovarnú kanvicu. Naplníme kanvicu
vodou, zapojíme do zásuvky a zapneme. Kým voda nevrie, čakáme.  Do pohára
vložíme vrecúško čaju a zalejeme vriacou vodou. Kanvicu vypneme a vypojíme zo
zásuvky. Počkáme 8 minút a potom vyberieme vrecúško.}

Super, vieme uvariť čaj (teda ak máme kanvicu atď..), ale v tejto chvíli nás
zaujímajú počítače. A počítač nevie naplniť kanvicu vodou, ani nevie, čo je
kanvica alebo čo je voda. Predošlý algoritmus je ozajstným \emph{algoritmom}
len vtedy, ak vieme, čo znamenajú jednotlivé úkony (napríklad čo znamená zobrať
pohár). 

\medskip

Keď budeme vytvárať algoritmy pre počítače, tak budeme používať celkom iné
inštrukcie -- také, ktorým počítač rozumie a vie ich aj vykonávať.

Medzi inštrukcie, ktoré počítač vie vykonávať patria napríklad jednoduché
aritmetické operácie (sčítanie, delenie\dots), čítanie z pamäte, ukladanie do
pamäte, ale aj mnoho ďalších. 

Avšak aby im dobre rozumel, musia byť zapísané ako hromada núl a jednotiek. A
tým zase veľmi nerozumieme my ľudia.  Preto existujú \textbf{Programovacie
jazyky}. Tie sú (väčšinou) dobre čitateľné ľuďmi a navyše si ich vie počítač
prepísať do svojich jednotiek a núl.

My sa budeme zaoberať programovacím jazykom C++, ktorý patrí medzi vyššie
programovacie jazyky -- to znamená, že sa lepšie číta ľuďmi a zvláda toho
oveľa viac ako nižšie.

\medskip

\textbf{Program} je skupina inštrukcií v nejakom programovacom jazyku.  A tieto
programy obvykle niečo robia, inak by nám boli nanič. 

V našom prípade budeme pracovať s programami, ktoré načítajú nejaký vstup a
vypíšu nejaký výstup.  Napríklad sčítací program môže načítať dve čísla a
vypísať ich súčet.

Napríklad toto:
\lstlang{cpp}\begin{lstlisting}
int main(){
}
\end{lstlisting}
je program v jazyku C++, ktorý nerobí vôbec nič. Ale dá sa skompilovať. 

\kapitola{Kompilácia - Linux}

\textbf{Kompilácia} je prepis programu napísaného v programovacom jazyku do
strojového kódu (jednotiek a núl), ktorý dokáže počítač spustiť.

Teraz sa ideme naučiť skompilovať program na Linuxe, vyskúšajte si to:
\begin{itemize}
    \item Uistíme sa, že máme nainštalovaný kompilátor (napríklad napíšeme \verb!g++ -v! malo by nám to 
    vypísať okrem balastu aj verziu kompilátora)
    \item Napíšeme program a uložíme ho do súboru \verb!hocico.cpp!
    \item Otvoríme konzolu a vojdeme do priečinka, kde sa nachádza súbor \verb!hocico.cpp!
    \item Skompilujeme jedným z nasledovných príkazov:
    \begin{itemize}
        \item \verb!g++ hocico.cpp! skompiluje program a výsledok uloží do \verb!a.out!
        \item \verb!g++ hocico.cpp -o hocico! skompiluje program a výsledok uloží do \verb!hocico!
        \item \verb!make hocico! za bežných okolností zavolá \verb!g++ hocico.cpp -o hocico!, ale dá
        sa nastaviť aj iné správanie
    \end{itemize}
    \item Ak kompilátor vypísal chybu, skúsime ju opraviť a skompilujeme
    program znova (nezabudnúť uložiť súbor) 
    \item Spustíme skompilovaný program
    buď \verb!./a.out! alebo \verb!./hocico!, podľa toho kam sa nám uložil.
    
\end{itemize}

\kapitola{Kompilácia - Windows}

\textbf{Kompilácia} je prepis programu napísaného v programovacom jazyku do
strojového kódu (jednotek a núl), ktorý dokáže počítač spustiť.

\kapitola{Náš prvý program}

Napíšme si náš prvý program a vysvetlíme si, čo robí:
\lstlang{cpp}\begin{lstlisting}
#include<cstdio>

int main(){
    printf("jesko\n");
}
\end{lstlisting}

V každom programe C++ sa musí nachádzať funkcia int main(). Táto funkcia sa
totiž začne vykonávať pri štarte programu, teda začnú sa vykonávať príkazy v
jej tele.  Jej telo je obalené kučeravými zátvorkami \verb'{' a \verb'}'. A
príkaz vo vnútri je len jeden, konkrétne funkcia \verb!printf("jesko\n");! (Za
každým príkazom musí nasledovať bodkočiarka.)

\medskip

Pokiaľ nevieme, čo sú to funkcie, môžeme si ich predstavovať ako samostatné
programy. Napríklad pokiaľ funkcia \verb!kup_mlieko()! by bola program na
kúpenie mlieka, tak algoritmus \verb!kup_mlieko();! \verb!kup_mlieko();!
\verb!kup_mlieko();! by kúpil tri mlieka.

Funkcie môžu mať aj argumenty/parametre, ktoré sa píšu do vnútra zátvoriek a
ovplyvňujú dianie funkcií. Môžete hádať, čo by urobil nasledovný algoritmus:
\verb!kup(mlieko);! \verb!kup(chlieb, tesco);! \verb!kup(rozky, kaufland);!.

Viac o funkciách sa naučíme neskôr, naučíme sa ich aj vytvárať.

\medskip

Vráťme sa k funkcii s názvom \verb!printf!. Túto funkciu normálne kompilátor
nepozná.  Preto je na začiatku programu riadok \verb!#include<cstdio>!.

\verb!cstdio! (c++ standard input output) je názov knižnice, v ktorej je
definovaných mnoho užitočných funkcií súvisiacich s načítavaním vstupu a
vypisovaním výstupu. Keď na začiatok programu dáme správny riadok, tak sa z
knižnice kompilátor naučí všetky pojmy.

Funkcia \verb!printf!, ktorá je v knižnici \verb!cstdio!, slúži na vypisovanie
výstupu.  A pokiaľ jej do zátvorky vložíme nejaký reťazec znakov v
úvodzovkách, tak ich vypíše na obrazovku.  V našom príklade má reťazec 6 znakov
\verb!'j'!, \verb!'e'!, \verb!'s'!, \verb!'k'!, \verb!'o'! a \verb!'\n'!.
Prvých 5 sú obyčajné písmená, posledný je znak konca riadku. (Niektoré znaky
(napríklad koniec riadku, alebo \verb!"!) nemôžeme do reťazca napísať priamo,
takže ich píšeme pomocou znaku \verb!'\'!. Viac nájdete napríklad na
\url{http://en.cppreference.com/w/cpp/language/escape})

\medskip

\cvicenie Skompilujte si vyššie uvedený program. A spustite ho. Čo sa stalo?
Čo sa stane, keď z programu vynecháte znak \verb!'\'!? Prečo?  Ako by ste
vypísali dva riadky so slovom \verb!jesko!?

\cvicenie Čo všetko môžeme z programu vynechať, aby stále robil to isté?
 
\cvicenie Skúste na obrazovku vypísať reťazec \verb!"jesko"! aj s
úvodzovkami.

\riesenie Program vypíše reťazec \verb"jesko". Keď vynecháme znak \verb!'\'!
program vypíše \verb"jeskon" a riadok neukončí.  Dvakrát vypísať reťazec
\verb"jesko" môžeme dvoma spôsobmi: buď \verb!printf("jesko\njesko\n");! alebo
\verb!printf("jesko\n"); printf("jesko\n");!.

\riesenie Niektoré prebytočné medzery a prázdny riadok. Z nasledovného
programu už nie je možné nič vynechať bez straty funkčnosti.
\lstlang{cpp}\begin{lstlisting}
#include<cstdio>
int main(){printf("jesko\n");}
\end{lstlisting}

\riesenie Použijeme príkaz \verb!printf("\"jesko\"\n");!

\kapitola{Premenné}
Bez toho, aby sme si mohli údaje zapamätávať sa programovať nedá. Na zapamätanie
nejakých hodnôt nám slúžia premenné. \textbf{Premennú} si môžeme predstaviť ako
krabičku, do ktorej vieme všeličo strčiť. Krabičky majú svoje meno a aj svoj
typ. 

Premennú s menom si môžeme predstaviť ako krabičku s nálepkou.
\emph{Meno} krabičky/premennej slúži na to, aby sme s nimi vedeli pracovať.
Napríklad "ulož číslo do tamtej premennej" počítač nepochopí, on by rád počul
"ulož číslo do premennej s menom \verb'a'". 

\emph{Typ} premennej zase určuje, čo do nej môžeme vkladať. Do premennej typu
\verb!int! (odvodené od anglického slova integer) môžeme ukladať celé číslo, do
premennej typu \verb!string! (z angličtiny sa dá preložiť ako reťazec) zasa môžeme vkladať reťazce
znakov (na \verb!string!y potrebujeme knižnicu). Neskôr sa naučíme používať
oveľa viac typov premenných, dokonca vyrábať vlastné typy.

\medskip

Ako premenné používať? Na to, aby sme mohli s premennou niečo robiť, si
ju musíme vytvoriť (deklarovať). To sa robí príkazom \verb!<typ-premennej> <meno-premennej>;!
napríklad \verb!int cislo;! 

Do premenných môžeme vkladať priradením, teda pomocou \verb'='. Syntax je
\verb!<meno-premennej> = <výraz>!, napríklad \verb!a = 4!. O výrazoch si
povieme viac neskôr, dôležité zatiaľ je, že výraz musí mať nejakú hodnotu.
Napríklad výraz $4$ má hodnotu $4$. Výraz $a$ má hodnotu toho, čo je vnútri
premennej/krabičky s menom $a$.  Výraz $4 + 7$ má hodnotu $11$. Pokiaľ v $a$ je
číslo $5$, tak výraz $a + 5$ má hodnotu $10$.

Výraz musí mať hodnotu preto, lebo práve táto hodnota sa uloží do premennej:
Napríklad keď napíšeme $a = 3 + 8;$ tak do a sa priradí číslo $11$.

\cvicenie Skúste určiť, čo robí nasledovný program. Aké hodnoty budú na
konci v premenných $a$, $b$ ,$c$?
\lstlang{cpp}\begin{lstlisting}
int main(){
    int a;
    a = 4;
    a = 7;

    int b, c;
    b = a;
    c = a + b;
    a = a + 4;
}
\end{lstlisting}

\riesenie
Najskôr sme vyrobili premennú $a$. Potom sme do nej priradili hodnotu $4$.
Následne sme do nej priradili hodnotu $7$ (stará hodnota $4$ sa zahodila), teda
premenná bude obsahovať iba číslo $7$. Potom sme vyrobili dve nové premenné $b$
a $c$ (správne, premenných rovnakého typu vieme vyrábať aj viac naraz, stačí
ich mená oddeliť čiarkou).  Do premennej $b$ sa priradila hodnota výrazu $a$,
teda $7$. Do $c$ sa priradí hodnota výrazu $a + b$, čo je hodnotou výrazu $7 +
7$ teda $14$. Nakoniec sa do $a$ priradí $a + 4$, čiže $7+4$, čiže $11$. V
$a,b,c$ budú na konci programu postupne hodnoty $11,7,14$.

\medskip

Aby sme vedeli lepšie pozorovať, čo sa v programe deje, naučíme sa vypisovať
premenné na obrazovku. Bude na to slúžiť príkaz \verb!printf()!, ale teraz bude
použitý zložitejšie. Funkcii \verb!printf()! totiž môžeme do zátvoriek napísať
viac argumentov ako jeden. Prvý argument totiž nemusí byť obyčajný reťazec,
ale môže byť špeciálny formátovací reťazec. Od obyčajného sa líši tým, že
obsahuje podivné značky ako \verb!%d! \verb!%lf! \verb!%4d! a mnohé ďalšie.
Tieto značky sú pri spracovaní funkciou nahradené hodnotami, ktoré určíme v
ďalších argumentoch tejto funkcie.

Vyskúšajte si skompilovať a spustiť nasledovný program.
\lstlang{cpp}\begin{lstlisting}
#include<cstdio>
int main(){
    printf("%d + %d = %d\n",4,7,4+7);
}
\end{lstlisting}
Mal by vypísať reťazec \verb!4 + 7 = 11!. Totiž funkcii \verb!printf! sme dali
4 argumenty (argumenty oddeľujeme čiarkou) reťazec \verb!"%d + %d = %d\n"! a
tri výrazy $4$, $7$ a $4+7$ s hodnotami $4$, $7$ a $11$.  Funkcia najprv
zoberie prvý argument, ktorý musí byť reťazec, nájde tam všetky výskyty
\verb!%<nieco>! a nahradí ich hodnotami ďalších argumentov.  \verb!%d! znamená,
že argument bude celé číslo, ktoré chceme vypísať v desiatkovej sústave. O
iných značkách si povieme viac neskôr, slúžia buď na vypisovanie iných typov
objektov (reálne čísla, znaky, reťazce\dots) alebo rôzny spôsob ich vypísania
(počet desatinných miest, sústava\dots).

Keďže premenné sú tiež výrazy, tak môžeme vypisovať ich hodnoty takto:
\lstlang{cpp}\begin{lstlisting}
#include<cstdio>
int main(){
    int a;
    a = 17;
    printf("premenna a ma hodnotu %d\n",a);
}
\end{lstlisting}
Alebo bez zbytočných blábolov:
\lstlang{cpp}\begin{lstlisting}
#include<cstdio>
int main(){
    int a = 7;
    printf("%d\n",a);
}
\end{lstlisting}
Všimnite si riadok \verb!int a = 7;!. Ide o skrátený zápis dvoch príkazov
\verb!int a; a = 7!. Dá sa to použiť aj pri deklarácií viacerých premenných,
teda \verb!int a = 1, b = 2, c = 3;! je skrátený zápis pre \verb!int a = 1;!
\verb!int b = 2;! \verb!int c = 3;!

\cvicenie Skúste na rôznych miestach programu z cvičenia 1 vypísať hodnoty
premenných. Všelijako upravujte program a pozorujte, čo sa deje.
\fakeriesenie

\nadpis{Typy číselných premenných}

K číselným premenným treba poznamenať ešte niekoľko dôležitých vecí.  Každá
číselná premenná a aj každé číslo má obmedzenú veľkosť.

Napríklad premenná typu \verb!int! môže mať v sebe uložené len celé číslo od
-2147483648 po 2147483647. Je to preto, že \verb!int! je v pamäti zapísaný ako
32 jednotiek a núl. Teda si vieme zapamätať len $2^{32}$ rôznych hodnôt. V
prípade premennej typu \verb!int! sú to hodnoty od $-2^{31}$ po $2^{31}-1$
(záporných čísel je presne polovica, kladných čísel je o 1 menej, kvôli číslu
0).

\cvicenie Vyskúšajte si spustiť príkaz \verb!printf("%d\n", 2147483647 + 1);!.
\fakeriesenie

\medskip

Existujú premenné aj s menším rozsahom, napríklad \verb!short int! má rozsah
$-32768..32767$ alebo $-2^{15}..2^{15}-1$.  \verb!char! má rozsah $-128..127$
resp. $-2^7..2^7-1$. Premenná typu \verb!bool! má len dve hodnoty 
\verb!false! resp. 0 (nepravda) alebo \verb!true! resp. 1 (pravda).

Väčší rozsah majú zasa premenné \verb!long long!, od $-2^{63}..2^{63}$. Pri ich
vypisovaní však treba dávať pozor, namiesto \verb!"%d"! treba písať
\verb!"%lld"! na Linuxe alebo \verb!"%I64d"! na Windowse.

Existujú aj typy premenných, ktoré neukladajú záporné čísla. Napríklad
\verb!unsigned int! má rozsah $0..2^{32}-1$.  Pri ich používaní treba byť
obozretný, lebo sa vám môže stať, že mínus jedna je viac ako nula.  (Totiž mínus
v bezznamienkovej premennej môže byť 4294967295.)

\nadpis{Binárna sústava a reprezentácia dát v pamäti.}

Každá hodnota je v pamäti počítača reprezentovaná ako postupnosť núl a
jednotiek -- bitov. Osem bitov je jeden bajt. Napríklad \verb!int! má veľkosť 4
bajty, čiže 32 bitov. \verb!char! má zase len jeden bajt -- 8 bitov. Keď máme v
premennej typu \verb!char! uložené číslo 47, tak v pamäti počítača to vyzerá
ako 00101111. Rovnaké číslo v premennej typu \verb!int! vyzerá 00000000
00000000 00000000 00101111. 

\textit{Poznámka, niektoré počítače alebo systémy majú iné poradie bitov v
pamäti.}
%\textit{Poznámka: niektoré počítače alebo systémy majú opačné poradie, teda 47
%môže vyzerať aj 00101111 00000000 00000000 00000000 ale aj 111101000 00000000
%00000000 00000000 alebo 00000000 00000000 00000000 00101111. My budeme
%pracovať s takým zápisom, na aký sme zvyknutý z desiatkovej sústavy. (Teda
%číslo tristodvadsaťpäť píšeme 325 nie 523)}

\medskip
Čísla sú teda v pamäti napísané v dvojkovej (binárnej) sústave. Tá je veľmi
podobná našej desiatkovej (decimálnej), takže sa jej netreba báť.  Aby sme sa
neplietli, tak teraz chvíľu budeme písať sústavu do dolného indexu, lebo napr.
$11_2 \neq 11_{10}$. (Alebo $3_{10} \neq 11_{10}$)

Keď v desiatkovej sústave napíšeme $325_{10}$ myslíme tým tri stovky, dve
desiatky a päť jednotiek. V dvojkovej sústave napr. $1101_2$ je jedna osmička,
jedna štvorka, žiadna dvojka a jedna jednotka. Dokopy $8+4+0+1 = 13$, teda
$1101_2 = 13_{10}$. 

Matematicky zapísané, $325_{10} = 3\cdot 100 + 2\cdot 10 + 5\cdot 1$. Ak chceme
napísať $325_{10}$ v dvojkovej sústave, musíme použiť namiesto mocnín desiatky
($1$, $10$, $100$, $1000$, \dots) mocniny dvojky ($1$, $2$, $4$, $8$, $16$,
\dots). Teda $325 = 1\cdot 256 + 0\cdot 128 + 1\cdot 64 + 0\cdot 32 + 0\cdot 16
+ 0\cdot 8 + 1\cdot 4 + 0\cdot 2 + 1\cdot 1$.  Preto $325_{10} = 101000101_2$.

Opačný prevod sme už zvládli vyššie, napríklad ak nás zaujíma hodnota
$101111_2$, stačí jednoducho sčítať $1\cdot 1 + 1\cdot 2 + 1\cdot 4 + 1\cdot 8
+ 0\cdot 16 + 1\cdot 32 = 47_{10}$.

\cvicenie Preveďte z desiatkovej sústavy do dvojkovej nasledovné čísla $31$, $200$,
$1024$.

\cvicenie Preveďte z dvojkovej sústavy do desiatkovej čísla $10101$,
$100000000$, $1100110011$.

\cvicenie Sčítajte v binárnej sústave čísla $110110$ a $1111101$. Následne
ich preveďte do desiatkovej sústavy a sčítajte v desiatkovej sústave. Výsledok
preveďte naspäť do dvojkovej a výsledok overte.

\riesenie $31_{10} = 11111_2$. $200_{10} = 11001000$. $1024_{10} =
10000000000$, lebo $1024 = 2^{10}$.

\riesenie $10101_2 = 21_{10}$. $100000000_2 = 256$. $1100110011_2 =
819_{10}$

\riesenie V binárnej sústave sčítavame rovnako ako v desiatkovej, napíšeme si
čísla pod seba, sčitavame zľava doprava a pripočítavame prípadné zvyšky.
\begin{verbatim}
    110110     =     54
+  1111101     =  + 125
   -------          ---
  10110011     =    179
\end{verbatim}

\medskip

Kladné čísla sú v pamäti reprezentované práve v dvojkovej sústave. Keď je číslo
príliš veľké, jednoducho sa zahodia prvé bity -- ukážeme si to na príklade.
Premenná typu \verb!unsigned char! má jeden bajt (8 bitov) a teda je schopná
reprezentovať čísla od 0 po 255. Nie je žiadným prekvapením, že 5 bude
zapamätané ako $00000101$, či $255$ ako $11111111$. Čo sa stane, keď ku $255$
pripočítame 1? Dostaneme $11111111 + 1 = 100000000$.  Keďže si môžeme zapamätať
len 8 bitov, musíme prvú jednotku zahodiť. Dostaneme $00000000$, čo je 0.

Preto pri premennej unsigned char $255+1 = 0$ alebo $100\cdot 100 = 16$.

\medskip

Ale ako je to s premennými ktoré podporujú aj záporné čísla?  Kladné čísla sa
reprezentujú tak ako predtým, napríklad v premennej typu \verb!char! $0 =
00000000$ až $127 = 01111111$.

Čo sa stane, keď od nuly odpočítame 1? $00000000 - 1 = ...11111111111111$
(nekonečné čislo) ale nás zaujíma len posledných 8 bitov, čo je $11111111$.
Takže $-1 = 11111111$. Keď ďalej odpočítavame 1 dostávame $-2 = 11111110$, $-3
= 11111101$ \dots $-128 = 10000000$. 

Super je, že sa nám zachovali logické vlastnosti, napríklad že $-1 + 1 = 0$,
$-50 + 40 = -10$, $-31 + 100 = 69$\dots bez toho, aby sme museli pri práci s
binárnym čislom rozlišovať, či je číslo záporné alebo kladné. Jednoducho
sčítame čísla pod sebou a ak sa výsledok nezmestí, tak zahodíme prvé bity.

Napr. $-1 + (-1) = 11111111 + 11111111 = 1011111110 \rightarrow 11111110 = -2$.

Tento spôsob reprezentovania záporných čísel sa nazýva aj dvojkový doplnkový
kód.

\textit{Vtip: "Ľudí môžeme rozdeliť do 10 skupín -- tých, ktorí rozumejú
binárnej sústave a tých, ktorí jej nerozumejú."}

\nadpis{Pretypovanie}

Sčítavať, odčítavať, násobiť, priraďovať, porovnávať\dots sa dajú len premenné
a čísla rovnakých typov. (Všimnite si, že aj čísla (napríklad číslo 45) majú
svoje typy.)

Pokiaľ sú pri operácii typy rôzne, treba ich pretypovať a naštastie v
C++ sa mnohé pretypovanie robí automaticky. Napríklad medzi číslami.  
\verb!int a = 47; char c = a;! nie je žiaden veľký problém.  (V tajnosti
prebehne jedno pretypovanie, v príkaze \verb!c = a!, sa a vyhodnotí ako
\verb!(int)47!, pretypuje sa na \verb!(char)47! a následne sa priradí i
\verb!c = (char)47!.

Pokiaľ pretypujeme z menších premenných do väčších, číslo sa doplní nulami
zľava.  \verb!(char)19! $= 00010011 \rightarrow 00000000 00000000 00000000
00010011 =$ \verb!(int)19!

Z väščích do menších, sa zasa zahadzujú prvé bity -- \textbf{pozor} pri tomto
pretypovaní sa môže číslo zmeniť.  \verb!(int)200! $= 00000000 00000000
00000000 11001000 \rightarrow 11001000 =$ \verb!(char)(-56)!  Tento jav sa volá
\textbf{pretečenie}.

\medskip

Bežné čísla, ktoré napíšete v programe majú typ \verb!int!.  Teda aj pri
\verb!char a = 20;! nastáva pretypovanie z \verb!int! do \verb!char!.

Keď niekde napíšete číslo $123\,456\,789\,123$, tak vám pravdepodobne kompilátor
vypíše varovanie, lebo dané číslo je typu \verb!int!, ale nezmestí sa do jeho
rozsahu. Treba písať \verb!123465789123LL! alebo \verb!123456789123ll!.

\medskip

Pri operáciach ako sčítanie, násobenie\dots na rôznych typoch sa najprv obe
čísla pretypujú na všeobecnejší typ, až potom sa spočítajú.

Napríklad ak \verb!char a = 20;! a \verb!int b = 20;! tak výsledkom \verb!a * b!
bude 400.

Treba si dávať \textbf{pozor} na nasledovnú vec.

\cvicenie Skopírujte, skompilujte a spustite nasledovný program.  (Pokiaľ
používate windows, použite miesto \verb!%lld! \verb!%I64d!)
\lstlang{cpp}\begin{lstlisting}
#include<cstdio>
int main(){
    ll a = 1000000*1000000;    
    printf("d = %lld\n",d);
}
\end{lstlisting}
Čo vypísal? Prečo? Ako by ste ho opravili, aby naozaj vypísal
$1\,000\,000\,000\,000$?

\riesenie Vynásobili sme dva \verb!int!y, keďže majú rovnaký typ, nič sa
nepretypuje.  Keďže výsledok je príliš veľký, nastáva pretečenie a výsledkom
násobenia bude \verb!int!  $-727\,379\,968$. Následne sa vykoná priradenie
\verb!a = -727379968!, ale správna hodnota už je nenávratne preč. Opraviť to
môžeme tak, že zmeníme \verb!1000000! na \verb!1000000LL!. (Stačí zmeniť
jedno.)

\medskip

Pretypovanie bežne funguje automaticky, ale vieme ho spôsobovať aj ručne.
Syntax je \verb!(<typ>)<hodnota>! alebo \verb!<typ>(hodnota)! alebo použijeme
obe dvojice zátvoriek.

\cvicenie Aké budu hodnoty v $a$, $b$ po nasledovných priradeniach? 
\begin{lstlisting}
int a = 1234, b;
b = (char)a;
b = char(a);
b = char(a)*a;
\end{lstlisting}

\riesenie Hodnota $a$ bude vždy 1234, pretypovanie nemení premenné. 
$b$ bude mať hodnoty -46, -46 a -56764 (teda $-46\cdot 1234$).

\medskip

\textbf{Poznámka}, \verb!bool! nie je \textit{číselný} typ. Teda \verb!bool(int(4))! nie
je 0 hoci posledný bit v zápise 4 je 0.  Pre \verb!bool! platí, že
\verb!bool(0)! je \verb!false! (resp. 0), všetky ostatné čísla sa zmenia na
\verb!true! (resp. 1).


\kapitola{Alternatívny výstup}

V C++ sa dá na výstup vypisovať aj inak ako cez \verb!printf!. Pokiaľ
includneme knižnicu iostream, môžeme vypisovať pomocou streamu
\verb!std::cout!. Používa sa veľmi jednoducho, hoci vo vnútri je schovanej veľa
mágie.

\begin{lstlisting}
#include<iostream>
int main(){
    std::cout << "Ahoj svet\n";
    int a = 4;
    std::cout << "a = " << a << endl;
}
\end{lstlisting}

Nevýhodou streamov je, že sú o máličko pomalšie ako príkazy z \verb!cstdio!. Bežne
to nepostrehnete, ale pokiaľ by ste chceli načítať/vypísať niekoľkostotisíc
prvkov, odporúčame použiť \verb!cstdio!.

Jednou z výhod je, že sa nemusíte starať o typy a formátovacie značky
\verb"%lld". Preto vám odporúčame používať streamy, ak programujete na Windowse
a vypisujete \verb!long long!y.

Pokiaľ nechcete stále písať pred každým príkazom \verb!std::!, stačí pridať
riadok \verb!using namespace std;!

\begin{lstlisting}
#include<iostream>
using namespace std;
int main(){
    cout << "Ahoj svet\n";
    int a = 4;
    cout << "a = " << a << endl;
}
\end{lstlisting}

\cvicenie Vezmite nejaký program, kde ste viackrát použili \verb!printf! a
prepíšte ho na streamy.
\fakeriesenie

\end{document}


