%Tu si môžete zaznačiť, že pracujete na danej veci. V prípade, že ste napísali len časť a ďalej už
%nechcete, alebo ste hotoví tak sa odtiaľ odpíšte. Bolo by však fajn, aby jedu vec robil jeden
%človek ak celok a zvyšný len kontrolovali
%vypracuva: Zaba 

%gram. oprava Maja
\input ../../include/include.tex

\begin{document}

\velkynadpis{Použitie hrubej sily}

Obsah:
\begin{itemize}
    \item cieľ tejto časti
    \item n!
    \item $2^n$
    \item orezávania
\end{itemize}


Úlohy čo tu chcem využit:

Máme číslo $n$. Nájdite najmenšie číslo menšie ako $n$, ktoré má všetky cifry rôzne.

Máme $n$ čísiel kladných aj záporných. Koľko je takých poradí preskúmavania, aby hodnota nikdy
neklesla pod 0.

Koniec úloh

%M: potialto som nekontrolovala, vymaze sa to, nie?

Ako sa vraví: keď to nejde silou, používate jej primálo.

Častokrát sa nám stane, že nevieme vymyslieť rýchle riešenie, jediné, čo nás napadne je hrubé
priamočiare skúšanie všetkých možností. Napríklad ak hľadáme všetky čísla menšie ako $1000$, ktoré
sú deliteľné $5$ a ich druhá mocnina je deliteľná $4$. Jedna možnosť je zamyslieť sa, ako také čísla
vyzerajú a vypísať tie správne. Druhá možnosť je naozaj vyskúšať všetkých $1000$ čísel, či
nevyhovujú zadaniu.

Takéto riešenie je pre nás ľudí nevyhovujúce. Umocňovať $1000$ čísel na druhú a ešte kontrolovať, či
sú deliteľné $4$? V žiadnom prípade. Ale počítač je na tom inak. Jeho hlavná výhoda je výpočtová
sila. Počítač nemá problém umocniť číslo a zistiť deliteľnosť v priebehu milisekúnd. A vyskúšať to
$1000$ krát \dots brnkačka.

V tejto kapitole sa teda porozprávame o tom, ako dostať silu počítača pod našu kontrolu, naučíme sa kedy a
ako používať hrubú silu a zopár trikov, aby to ani tomu počítaču netrvalo tak dlho. A ak sa nudíte,
môžete sa rovno vrhnúť na úlohu \textbf{magic}.

\textbf{Permutácie}

Prvým spôsob ako generovať všetky možnosti sú permutácie. Zoberme si napríklad úlohu
\textbf{pamiatky}. V tejto úlohe máme $n$ pamiatok a na svojej výhliadkovej túre si ich chceme
všetky postupne pozrieť. Každá pamiatka nám však pridá alebo uberie radosť z výletu. Samozrejme
nechceme, aby naša radosť bola pod $0$. Otázka je, koľko je takých postupností návštevy pamiatok, že
naša radosť neklesne pod $0$.

Vieme, že existuje najviac $n!$ rôznych poradí návštev pamiatok, keďže každá permutácia pamiatok určuje jednu
postupnosť, ktorou sa môžeme vybrať. Takisto vieme, že v najväčšom možnom vstupe je $n=9$, takže
rôznych permutácií je $9!=362\,880$. A to vôbec nie je tak veľa.

No a úplne najlepšia správa je, že ak vám dám konkrétnu permutáciu pamiatok, viete veľmi jednoducho
zistiť, či táto permutácia poruší podmienku zo zadania a teda radosť z jej prejdenia klesne pod $0$.
Na to stačí simulovať prechádzku po pamiatkach a stále si pamätať, akú máme momentálne radosť. Ak
niekedy klesne pod $0$, daná postupnosť je zlá, ak sa to nikdy nestane, môžeme si pripočítať $1$ k
výsledku.

Ten najdôležitejší krok však zostáva. Ako si vygenerovať postupne všetky permutácie čísel od $1$ po
$n$. Poprípade to ani nemusia byť prvky od $1$ po $n$, môžu to byť aj čísla oveľa väčšie alebo
opakujúce sa.

\medskip

Prvou možnosťou je urobiť si vlastnú rekurzívnu funkciu. Ako parametre dostane táto funkcia dve
polia: prvé obsahuje prvky, ktoré ešte nepoužila a treba ich zaradiť do vytváranej permutácie, druhá
obsahuje prvky, ktoré vytvárajú aktuálnu permutáciu. To, čo treba spraviť ako ďalší krok, je vybrať
prvok, ktorý bude ležať na ďalšom mieste v našej permutácii. A vhodní kandidáti sú všetky ešte
nezaradené prvky. No a keďže chceme vytvoriť permutácie všetky, postupne skúšame dosadiť všetky
nezaradené prvky na ďalšie voľné miesto. Keď si zvolíme prvok $x$, vyberieme ho z nezaradených prvkov,
pridáme ho do permutácie a znova sa rekurzívne vnoríme.

Keď spotrebujeme všetky prvky, ostane nám výsledná permutácia. A tento prístup ich postupne
vygeneruje všetky. Tu je k tomu názorný program:

\listing{brute-force/permutacie-rekurzia.cpp}

Nevýhodou na tomto riešení je, že si posúva dve polia ako parametre funkcie. To ho dosť spomaľuje.
Samozrejme, dá sa toho zbaviť, ale napriek tomu potrebujeme napísať sami zbytočne veľa kódu, v
ktorom je veľká šanca, že sa pomýlime. Niekto však ten kód už napísal za nás, tak prečo ho nepoužiť.

V STL existuje funkcia \texttt{next\_permutation()}, ktorá ako napovedá názov vytvorí z poľa, ktoré
jej dáme ako parameter ďalšiu jeho permutáciu. A toto má množstvo výhod. Nemusíme skoro nič
programovať, všetko dostaneme zadarmo. Túto funkcia môžeme používať opakovane a prejsť tak postupne
cez všetky možné permutácie. A taktiež vždy nájde permutáciu, ktorá je lexikograficky väčšia. To
znamená, že ak pole obsahuje niekoľko rovnakých prvkov, táto funkcia vie preskočiť všetky také,
ktoré sa líšia len výmenou niektorých dvoch rovnakých. Ak teda máme $n$ prvkov a $p$ z nich sú
rovnaké a ostatné sú rôzne, tak namiesto $n!$ možností vygeneruje len $\frac{n!}{p!}$.

Jediné, na čo si treba dať pozor je, že na začiatku chcete dať do tejto funkcie prvky usporiadané od
najmenšieho po najväčší. Toto je totiž lexikograficky najmenšia permutácia a ako som spomínal
\texttt{next\_permutation()} vytvára vždy väčšiu. Tu si môžete pozriež najčastejšie použitie tejto
funkcie:

\listing{brute-force/permutacie-next.cpp}

Bude veľmi dobré, ak sa tento kúsok kódu naučíte, zíde sa vám, keď budete potrebovať generovať všetky
permutácie.

\nadpis{Množiny}

Množina je ďalší matematický objekt, ktorý nám vie pomôcť. Zoberme si úlohu \textbf{zlodej}. V tejto
úlohe ste zlodej, ktorý sa snaží z domu ukradnúť predmety s čo najväčšou spoločnou váhou, má však
batoh, ktorý unesie najviac hmotnosť $w$. Chcete teda vybrať takú sadu vecí, že ich spoločná váha je
čo najbližšie k $w$.

Táto úloha je od predchádzajúcej trochu odlišná. Je pravda, že by sme sa stále mohli pýtať na všetky
permutácie prvkov, pridávať ich v tom poradí do batohu a keď batoh naplníme, prestaneme a zrátame,
koľko vážia dokopy veci v batohu.

To však ani zďaleka nie je to, čo by sme chceli robiť, dokonca to od nás nevyžaduje ani úloha. Nás
totiž vôbec nezaujíma, v akom poradí vložíme veci do batoha. Nás zaujíma, ktoré veci budú na konci v
batohu. Predstavte si, že viete vložiť $k$ vecí do batohu. Existuje $k!$ možností ako ich tam
vložiť, ale stále dostaneme to isté -- tieto predmety sa do batohu zmestia a zaberú stále rovnako
miesta. Bolo by fajn teda vyskúšať túto množinu len raz.

A tu prichádzame na koreň problému -- množiny. Chceli by sme vedieť rýchlo prezerať všetky množiny
vecí (množina je reprezentovaná tým, či danú vec zoberiem alebo nie) a potom overiť, či daná
množina vyhovuje. Výhodou je, že všetkých množín z $n$ prvkov je $2^n$, čo je stále veľké číslo, ale
oproti $n!$ nám to umožňuje spracovať všetky množiny až z $25$ prvkov.
Opäť existuje niekoľko prístupov na riešenie tohoto problému.

Prvým je znova rekurzia. V tomto prípade opäť budeme mať množinu vecí, ktoré ešte chceme zadeliť a
množinu vecí, ktoré už patria do našej množiny. Na začiatku volania zoberieme prvú vec, ktorá je
voľná a rozhodneme sa, či ju zobrať alebo nie. Samozrejme, keďže sa snažíme nájsť všetky možnosti,
treba skúsiť obe rozhodnutia. Keď prvok vyberieme, zaradíme ho do množiny vecí, ktoré už máme, vyhodíme
ju z nerozhodnutých a rekurzívne sa zavoláme. Ak si ju neberieme, aj tak ju vyhodíme z
nerozhodnutých vecí, lebo sme sa už rozhodli, že ju neberieme a zavoláme sa rekurzívne. Dostaneme
takýto jednoduchý kúsok kódu:

\listing{brute-force/mnoziny-rekurzia.cpp}

Opäť to však nie je najľahší prístup. Existuje aj prístup oveľa jednoduchší, samozrejme, ale viac
trikovejší. Majme našich $n$ vecí a dajme im nejaké poradie, napríklad podľa toho, kedy sa
vyskytli na vstupe, nultý, prvý až $n-1$-vý. Každú množinu si teraz viem predstaviť ako reťazec dlhý
$n$ znakov, tvorených zo znakov $0$ a $1$. $1$ znamená, že prvok sa v tejto množine nachádza, $0$ je
opak. Dobre, ale toto je vlastne spôsob, akým sa zapisujú binárne čísla. Stačí $i$-temu prvku v
poradí pŕiradiť hodnotu $2^i$ a z našeho reťazca sa razom stane celé číslo. A presne toto bude
princíp našho riešenie.

Môžeme si uvedomiť, že čísla od $0$ po $2^n-1$ nám postupne v dvojkovej sústave kóduju všetky možné
množiny $n$ prvkov, začínajúc prázdnou a končiac takou, čo obsahuje všetky prvky. A s číslami sa v
C++ pracuje veľmi pohodlne a keďže sú interne reprezentované ako $0$ a $1$, tak s nimi vieme robiť
veľa pekných operácií. Tu je program, ktorý ukazuje, ako sa dá postupne prejsť cez všetky množiny a
tiež zistiť, ktoré čísla do nich patria:

\listing{brute-force/mnoziny-bit.cpp}

S niektorými operáciami ste sa možno ešte nestretli. \texttt{$1<<n$} je takzvaný bitový posun doľava. V
princípe to znamená, že k číslu naľavo ($1$) v bitovom zápise pridá číslo napravo ($n$) núl. Teda z
pôvodného čísla vznikne číslo $2^n$. Ďalšia neznáma je \texttt{$i\&(1<<j)$}. Už vieme, že sa vlastne pýtame na
to, že $i\&2^j$. Čo to ale je? Znak \verb'&' predstavuje bitový and, ktorý zoberie bitový zápis čísla
$i$, bitový zápis čísla $2^j$ a bit po bite vykoná and. Ako vieme, $1 and 1 = 1$ a $0 and hocičo =
0$. Číslo $2^j$ obsahuje jedinú jednotku na $j$-tom mieste, všade inde sú $0$. To znamená, že tento
výraz bude mať hodnotu $0$, ak sa na $j$-tom mieste v množine $i$ nachádza $0$ a hodnotu $2^j$, ak
sa tam nachádza $1$. A keďže v C++ $0$ je $false$ a všetko ostatné je $true$, podmienka bude
splnená práve vtedy, keď množina $i$ obsahuje $j$-ty prvok. Šikovné, nie?

\end{document}
