%Tu si môžete zaznačiť, že pracujete na danej veci. V prípade, že ste napísali len časť a ďalej už
%nechcete, alebo ste hotoví tak sa odtiaľ odpíšte. Bolo by však fajn, aby jedu vec robil jeden
%človek ak celok a zvyšný len kontrolovali
%vypracuva: Zaba 
\input ../../include/include.tex

\begin{document}

\velkynadpis{Použitie hrubej sily}

Obsah:
\begin{itemize}
    \item cieľ tejto časti
    \item n!
    \item $2^n$
    \item orezávania
\end{itemize}

Úlohy čo tu chcem využit:

Máme číslo $n$. Nájdite najmenšie číslo menšie ako $n$, ktoré má všetky cifry rôzne.

Máme $n$ čísiel kladných aj záporných. Koľko je takých poradí preskúmavania, aby hodnota nikdy
neklesla pod 0.

Koniec úloh

Ako sa vraví: Keď to nejde silou, používate jej primálo.

Častokrát sa nám stane, že nevieme vymyslieť rýchle riešenie, jediné čo nás napadne je hrubé
priamočiare skúšanie všetkých možností. Napríklad ak hľadáme všetky čísla menšie ako $1000$, ktoré
sú deliteľné $5$ a ich druhá mocnina je deliteľná $4$. Jedna možnosť je zamyslieť sa, ako také čísla
vyzerajú a vypísať tie správne. Druhá možnosť je naozaj vyskúšať všetkých $1000$ čísiel či
nevyhovujú zadaniu.

Takéto riešenie je pre nás ľudí nevyhovujúce. Umocňovať $1000$ čísiel na druhú a ešte kontrolovať či
sú deliteľné $4$? V žiadnom prípade. Ale počítač je na tom inak. Jeho hlavná výhoda je výpočtová
sila. Počítač nemá problém umocniť číslo a zistiť deliteľnosť v priebehu milisekúnd. A vyskúšať to
$1000$ krát \dots brnkačka.

V tejto kapitole sa teda porozprávame, ako dostať silu počítaču pod našu kontrolu, naučíme sa kedy a
ako používať hrubú silu a zopár trikov, aby to ani tomu počítaču netrvalo tak dlho. A ak sa nudíte,
môžete sa rovno vrhnúť na úlohu \textbf{rôzne cifry}.

\textbf{Permutácie}

Prvou možnosťou na generovanie všetkých možností sú permutácie. Zoberme si napríklad úlohu
\textbf{pamiatky}. V tejto úlohe máme $n$ pamiatok a na svojej výhliadkovej túre, si ich chceme
všetky postupne pozrieť. Každá pamiatka nám však pridá alebo uberie radosť z výletu. Samozrejme
nechceme, aby naša radosť bola pod $0$. Otázka je, koľko je takých postupností návštevy pamiatok, že
naša radosť neklesne pod $0$.

Vieme, že existuje najviac $n!$ rôznych pozretí pamiatok, keďže každá ich permutácia určuje jednu
postupnosť, ktorou sa môžeme vybrať. Takisto vieme, že v najväčšom možnom vstupe, je $n=9$, takže
rôznych permutácií je $9!=362\,880$. A to vôbec nie je tak veľa.

No a úplne najlepšia správa je, že ak vám dám konkrétnu permutáciu pamiatok, viete veľmi jednoducho
zistiť, či táto permutácia poruší podmienku zo zadania a teda radosť z jej prejdenia klesne pod $0$.
Na to stačí simulovať prechádzku po pamiatkach a stále si pamätať, akú máme momentálne radosť. Ak
niekedy klesne pod $0$, daná postupnosť je zlá, ak sa to nikdy nestane, môžeme si pripočítať $1$ k
výsledku.

Ten najdôležitejší krok však zostáva. Ako si vygenerovať postupne všetky permutácie čísiel od $1$ po
$n$. Poprípade to ani nemusia byť prvky od $1$ po $n$, môžu to byť aj čísla oveľa väčšie alebo
opakujúce sa.

\medskip

Prvou možnosťou je spraviť si vlastnú rekurzívnu funkciu. Ako parametre dostane táto funkcia dve
polia: prvé obsahuje prvky, ktoré ešte nepoužila a treba ich zaradiť do vytváranej permutácie, druhá
obsahuje prvky, ktoré vytvárajú aktuálnu permutáciu. To čo treba spraviť ako ďalší krok, je vybrať
prvok, ktorý bude ležať na ďalšom mieste v našej permutácii. A vhodný kandidáti sú všetky ešte
nezaradené prvky. No a keďže chceme vytvoriť permutácie všetky, postupne skúšame dosadiť všetky
nezaradené prvky na ďalšie voľné miesto. Keď si zvolíme prvok $x$, vyberieme ho z nezaradených prvkov,
pridáme ho do permutácie a znova sa rekurzívne vnoríme.

Keď spotrebujeme všetky prvky ostane nám výsledná permutácia. A tento prístup ich postupne
vygeneruje všetky. Tu je k tomu názorný program:

%\listing{permutacie-rekurzia.cpp}

Nevýhodou na tomto riešení je, že si posúva dve polia ako parametre funkcie. To ho dosť spomaľuje.
Samozrejme, dá sa toho zbaviť, ale napriek tomu, potrebujeme napísať sami zbytočne veľa kódu, v
ktorom je veľká šanca, že sa pomýlime. Niekto však ten kód už napísal za nás, tak prečo ho nepoužiť.

V STL existuje funkcia \texttt{next_permutation()}, ktorá ako napovedá názov vytvorí z poľa, ktoré
jej dáme ako parameter ďalšiu jeho permutáciu. A toto má množstvo výhod. Nemusíme skoro nič
programovať, všetko dostaneme zadarmo. Túto funkcia môžeme používať opakovane a prejsť tak postupne
cez všetky možné permutácie. A taktiež vždy nájde permutáciu, ktorá je lexikograficky väčšia. To
znamená, že ak pole obsahuje niekoľko rovnakých prvkov, táto funkcia vie preskočiť všetky také,
ktoré sa líšia len výmenou niektorých dvoch rovnakých. Ak teda máme $n$ prvkov a $p$ z nich sú
rovnaké a ostatné sú rôzne, tak namiesto $n!$ možností vygeneruje len $\fact{n!}{p!}$.

Jediné na čo si treba dať pozor je, že na začiatku chcete dať do tejto funkcie prvky usporiadané od
najmenšieho po najväčší. Toto je totiž lexikograficky najmenšia permutácia a ako som spomínal
\texttt{next_permutation()} vytvára vždy väčšiu. Tu si môžete pozriež najčastejšie použitie tejto
funkcie:

%\listing{permutacie-next.cpp}

Bude veľmi dobré, ak sa tento kúsok kódu naučíte, zíde sa vám, keď budete generovať všetky
permutácie.

\end{document}
