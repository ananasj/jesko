%Tu si môžete zaznačiť, že pracujete na danej veci. V prípade, že ste napísali len časť a ďalej už
%nechcete, alebo ste hotoví tak sa odtiaľ odpíšte. Bolo by však fajn, aby jedu vec robil jeden
%človek ak celok a zvyšný len kontrolovali
%vypracuva: Jano
\input ../../include/include.tex

\begin{document}

\velkynadpis{Úvod do C++}

\textit{Ako správne používať tento študijný text: Milý čitateľ. Chystáš sa prečítať si kuchárku o
základoch programovacieho jazyka C++. Najdôležitejšie pri učení nového programovacieho jazyka 
je poriadne si ho precvičiť. Preto počas čítania narazíš na niekoľko úloh, ktoré ti odporúčame vyriešiť a naprogramovať.
Má to dopomôcť k tomu, aby si všetko lepšie zapamätal a dostal do krvi.}

Obsah:
\begin{itemize}
    \item čo je to program?
    \item kompilácia - linux
    \item kompilácia - windows %TODO
    \item náš prvý program
    \item premenné
\end{itemize}

\medskip

\kapitola{Program}
\textit{O čom tu celý čas rozprávame? Čo je to program? Čo je to algoritmus? Čo je to programovací jazyk?
Ak si myslíš že tieto základy už vieš, môžeš túto sekciu preskočiť, inak si ju radšej prečítaj}

\textbf{Algoritmus} je potupnosť niekoľkých dobre definovaných inštrukcií -- úkonov, ktorá slúži na vykonanie nejakej úlohy.
Napríklad algoritmus na uvarenie čaju môže byť nasledovný:

\texttt{Zoberieme pohár, vrecúško čaju a rýchlovarnú kanvicu. Naplníme kanvicu vodou, zapojíme do zásuvky a zapneme. Kým voda nevrie, čakáme. 
Do pohára vložime vrecúško čaju a zalejeme vriacou vodou. Kanvicu vypneme a vypojíme zo zásuvky. Počkáme 8 minút a potom vyberieme vrecúško.}

Super, vieme uvariť čaj (teda ak máme kanvicu atď..) ale v tejto chvíli nás zaujímajú počítače. A počítač nevie naplniť kanvicu vodou,
ani nevie čo je kanvica alebo čo je voda. Predošlý algoritmus je ozajstným \emph{algoritmom} len vtedy, ak vieme čo znamenajú jednotlivé úkony 
(napríklad čo znamená zobrať pohár). 

\medskip

Keď budeme vytvárať algoritmy pre počítače, tak budeme používať celkom iné inštrukcie -- také ktorým počítač rozumie a vie ich aj vykonávať.

Medzi inštrukcie, ktoré počítač vie vykonávať patria napríklad jednoduché aritmetické operácie (sčítanie, delenie\dots), 
čítanie z pamäte ukladanie do pamäte ale aj mnoho ďalších. 

Avšak aby im dobre rozumel, musia byť zapísané ako hromada núl a jednotiek. A tým zasa veľmi nerozumieme my ľudia.
Preto existujú \textbf{Programovacie jazyky}. Tie sú (väčšinou) dobre čitateľné ľudmi a navyše si ich vie program prepísať
do svojich jednotiek a núl.

My sa budeme zaoberať programovacím jazykom C++, ktorý patrí medzi vyššie programovacie jazyky -- to znamená, že sa
lepšie čítajú ľuďmi a zvládajú toho oveľa viac nižšie.

\medskip

\textbf{Program} je skupina inštrukcií v nejakom programovacom jazyku.
A tieto programy obvykle niečo robia, inak by nám boli nanič. 

V našom prípade budeme pracovať s programami, ktoré načítajú nejaký vstup a vypíšu nejaký výstup.
Napríklad sčítací program môže načítať dve čísla a vypísať ich súčet.

Napríklad toto: \lstlang{cpp}\begin{lstlisting}
int main(){
}
\end{lstlisting}
je program v jazyku C++, ktorý nerobí vôbec nič. Ale dá sa skompilovať. 

\kapitola{Kompilácia - Linux}

\textbf{Kompilácia} je prepis programu napísaného v programovacom jazyku do strojového kódu (jednotek a núl),
ktorý dokáže počítač spustiť.

Teraz sa ideme naučiť skompilovať program na Linuxe, vyskúšajte si to:
\begin{itemize}
    \item Uistíme sa, že máme nainštalovaný kompilátor (napríklad napíšeme \verb!g++ -v! malo by nám to 
    výpísať okrem balastu aj verziu kompilátora)
    \item Napíšeme program a uložíme ho súboru \verb!hocico.cpp!
    \item Otvoríme konzolu a vôjdeme do priečinka, kde sa nachádza súbor \verb!hocico.cpp!
    \item Skompilujeme jendným z nasledovných príkazov:
    \begin{itemize}
        \item \verb!g++ hocico.cpp! skompiluje program a výsledok uloží do \verb!a.out!
        \item \verb!g++ hocico.cpp -o hocico! skompiluje program a výsledok uloží do \verb!hocico!
        \item \verb!make hocico! za bežných okolností zavolá \verb!g++ hocico.cpp -o hocico!, ale dá
        sa nastaviť aj iné správanie
    \end{itemize}
    \item Ak kompilátor vypísal chybu, skúsime ju opraviť a skompilujeme program znova (nezabudnúť uložiť súbor)
    \item Spustíme skompilovaný program buď \verb!./a.out! alebo \verb!./hocico!, podľa toho kam sa nám uložil.
    
\end{itemize}

\kapitola{Kompilácia - Windows}

\textbf{Kompilácia} je prepis programu napísaného v programovacom jazyku do strojového kódu (jednotek a núl),
ktorý dokáže počítač spustiť.

\kapitola{Náš prvý program}

Napíšme si náš prvý program a vysvetlime si, čo robí:
\lstlang{cpp}\begin{lstlisting}
#include<cstdio>

int main(){
    printf("jesko\n");
}
\end{lstlisting}

V každom programe C++ sa musí nachádzať funkcia int main(). Táto funkcia sa totiž začne 
vykonávať pri štarte programu, teda začnú sa vykonávať príkazy v jej tele.
Jej telo je obalené kučeravými zátvorkami \verb'{' a \verb'}'. A príkaz vo vnútri je len jeden, konkrétne funkcia
\verb!printf("jesko\n");! (Za každým príkazom musí nasledovať bodkočiarka.)

\medskip

Pokiaľ nevieme, čo sú to funkcie, môžeme si ich predstavovať ako samostatné programy. Napríklad pokiaľ
funkcia \verb!kup_mlieko()! by mohla kúpiť mlieko (len ju treba najprv naprogramovať). Keď by sme mali algoritmus:
\verb!kup_mlieko();! \verb!kup_mlieko();! \verb!kup_mlieko();! tak by sme dostali tri mlieka. 

Funkcie môžu mať aj argumenty/parametre, ktoré sa píšu do vnútra zátvoriek a ovplyvnňujú dianie funkcií. Môžete hádať, čo 
by urobil nasledovný algoritmus: \verb!kup(mlieko)! \verb!kup(chlieb, tesco)! \verb!kup(rozky, kaufland)!.

Viac o funkciách sa naučíme neskôr, naučíme sa ich aj vytvárať.

\medskip

Vráťme sa k funkcii s názvom \verb!printf!. Túto funkciu normálne kompilátor nepozná.
Preto je na začiatku programu riadok \verb!#include<cstdio>!.

\verb!cstdio! (c++ standard input output) je názov knižnice, v ktorej je definovaných mnoho užitočných funkcií
súvisiacich s načítavanim vstupu a vypisovaným výstupu. Keď na začiatok programu dáme správny riadok
tak sa z knižnice kompilátor naučí všetky pojmy.

Funkcia \verb!printf!, ktorá je v knižnici \verb!cstdio!, slúži na vypisovanie výstupu.
A pokiaľ jej do závtorky vopcháme nejaký reťazec znakov v úvodzovkách, tak ich vypíše na obrazovku.
V našom príklade má reťazec 6 znakov \verb!'j'!, \verb!'e'!, \verb!'s'!, \verb!'k'!, \verb!'o'! a \verb!'\n'!. 
Prvých 5 sú obyčajné písmená, posledný je znak konca riadku. (Niektoré znaky (napríklad koniec riadku, alebo \verb!"!) 
nemôžeme do reťazca napísať priamo, takže ich píšeme pomocou znaku \verb!'\'!. Viac nájdete napríklad na 
\url{http://en.cppreference.com/w/cpp/language/escape})

\medskip

\cvicenie{1} skompilujte si vyššie uvedený program. A spustite ho. Čo sa stalo? Čo sa stane, keď z programu vynecháte znak \verb!'\'!? Prečo?
Ako by ste vypísali dva riadky so slovom ješko?

\cvicenie{2} čo všetko môžeme s programu vynechať, aby stále robil to isté?

\cvicenie{3} skúste na obrazovku vypísať reťazec \verb!"jesko"! aj s úvodzovkami.

\riesenie{1} Program vypíše reťazec \verb"jesko". Keď vynecháme znak \verb!'\'! program vypíše \verb"jeskon" a riadok neukončí.
Dva krát vypísať reťazec \verb"jesko" môžeme dvoma spôsobmi: buď \verb!printf("jesko\njesko\n");! alebo \verb!printf("jesko\n"); printf("jesko\n");!.

\riesenie{2} Niektoré prebytočné medzery a prázdny riadok. Z nasledovného programu už nie je možné nič vynechať bez straty funkčnosti.
\lstlang{cpp}\begin{lstlisting}
#include<cstdio>
int main(){printf("jesko\n");}
\end{lstlisting}

\riesenie{3} Použijeme príkaz \verb!printf("\"jesko\"\n");!

\kapitola{Premenné}

Bez toho aby sme si mohli údaje zapamätávať sa programovať nedá. Na zapamätanie nejakých hodnôt nám slúžia premenné. \textbf{Premennú} 
si môžeme predstaviť ako krabičku, do ktorej vieme všeličo strčiť. Krabičky majú svoje mená a aj svoj typ. 

\emph{Meno} premennej slúži na to, aby sme s nimi vedeli pracovať. Napríklad "ulož číslo do tamtej premennej" počítač nepochopí, on by rád počul
"ulož číslo do premennej s menom 'krabicka'". 

\emph{Typ} premennej zasa určuje, čo do nej môžeme vkladať. Do premennej typu int (odvodené od anglického slova integer) môžeme ukladať celé číslo,
do premennej typu char (od slova character) zasa môžeme vkladať znaky. Neskôr sa naučíme používať oveľa viac typov premenných, dokonca vyrábať vlastné typy.

\medskip

Ako premenné používať? Na to aby sme mohli s premennou niečo robiť, musíme si nejakú vytvoriť (deklarovať). To sa robí príkazom
\verb!<typ-premennej> <meno-premennej>;! napríklad \verb!int cislo;! 

Do premenných môžeme vkladať priradením, teda pomocou \verb'='. Syntax je 
\verb!<meno-premennej> = <výraz>!, napríklad \verb!a = 4!. O výrazoch si povieme viac neskôr, dôležité zatiaľ je, že výraz musí mať nejakú hodnotu.
Napríklad výraz $4$ má hodnotu $4$. Výraz $a$ má hodnotu toho, čo je vnútri premennej/krabičky s menom $a$.
Výraz $4 + 7$ má hodnotu $11$. Pokiaľ v $a$ je číslo $5$, tak výraz $a + 5$ má hodnotu $10$.

Výraz musí mať hodnotu preto, lebo práve táto hodnota sa uloží do premennej: 
Napríklad, keď napíšeme $a = 3 + 8;$ tak do a sa priradí číslo $11$.

\cvicenie{1} Skúste určiť, čo robí nasledovný program. Aké hodnoty budu na konci v premenných $a$, $b$ ,$c$?
\lstlang{cpp}\begin{lstlisting}
int main(){
    int a;
    a = 4;
    a = 7;

    int b, c;
    b = a;
    c = a + b;
    a = a + 4;
}
\end{lstlisting}

\riesenie{1}
Najskôr sme vyrobili premennú $a$. Potom sme do nej priradili hodnotu $4$. Následne sme do nej priradili hodnotu $7$ (stará hodnota $4$ sa zahodila), teda premenná bude obsahovať
iba číslo $7$. Potom sme vyrobili dve nové premenné $b$ a $c$ (správne, premenných rovnakého typu vieme vyrábať aj viac naraz, stačí ich mená oddeliť čiarkou).
Do premennej $b$ sa priradila hodnota výrazu $a$, teda $7$. Do $c$ sa priradí hodnota výrazu $a + b$, čo je hodnotou výrazu $7 + 7$ teda $14$. Nakoniec sa do $a$ priradí
$a + 4$, čiže $7+4$, čiže $11$. V $a,b,c$ budú na konci programu postupne hodnoty $11,7,14$.

\medskip

Aby sme vedeli lepšie pozorovať, čo sa v programe deje, naučíme sa vypisovať premenné na obrazovku. Bude na to slúžiť príkaz
\verb!printf()!, ale teraz bude použitý zložitejšie. Funkcii \verb!printf()! totiž môžeme do zátvoriek napísať viac argumentov ako jeden.
Prvý parameter totiž nie je obyčajný reťazec, ale špecíálny formátovací reťazec. Od obyčajného sa lýši tým,
že obsahuje podivné značky ako \verb!%d! \verb!%lf! \verb!%4d! a mnohé ďalšie. Tieto značky sú pri spracovaní funkciou nahradené hodnotami, ktoré určíme v ďalších argumentoch tejto funkcie.

Vyskúšajte si skompilovať a spustiť nasledovný program.
\lstlang{cpp}\begin{lstlisting}
#include<cstdio>
int main(){
    printf("%d + %d = %d\n",4,7,4+7);
}
\end{lstlisting}
Mal by vypísať reťazec \verb!4 + 7 = 11!. Totiž funkcii printf sme dali 4 argumenty (argumenty oddelujeme čiarkou) reťazec \verb!"%d + %d = %d\n"! a tri výrazy $4$, $7$ a $4+7$ s hodnotami $4$, $7$ a $11$.
Funkcia najprv zoberie prvý argument, ktorý musí byť formátovací režazec, nájde tam všetky výskyty \verb!%<nieco>! a nahradí ich hodnotami ďalších argumentov.
\verb!%d! znamená, že argument bude celé číslo, ktoré chceme vypísať v desiatkovej sústave. O iných značkách si povieme viac neskôr, slúžia buď na vypisovanie
iných typov objektov (reálne čísla, znaky, reťazce\dots) alebo rôzny spôsob ich vypísania (počet desatinných miest, sústava\dots).

Keďže premenné sú tiež výrazy, tak môžeme vypisovať ich hodnoty takto:
\lstlang{cpp}\begin{lstlisting}
#include<cstdio>
int main(){
    int a;
    a = 17;
    printf("premenna a ma hodnotu %d\n",a);
}
\end{lstlisting}
Alebo bez zbytočných blábolov:
\lstlang{cpp}\begin{lstlisting}
#include<cstdio>
int main(){
    int a = 7;
    printf("%d\n",a);
}
\end{lstlisting}
Všimnite si riadok \verb!int a = 7;! ide o skrátený zápis dvoch príkazov \verb!int a; a = 7!. Dá sa to použiť aj pri deklarácií viacerých premenných
teda \verb!int a = 1, b = 2, c = 3;! je skrátený zápis pre \verb!int a = 1;! \verb!int b = 2;! \verb!int c = 3;!

\cvicenie{2} Skúste na rôznych miestach programu vypísať hodnoty premenných. Všeliako upravujte program a pozorujte, čo sa deje.





\end{document}


