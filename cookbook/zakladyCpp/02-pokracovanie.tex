%Tu si môžete zaznačiť, že pracujete na danej veci. V prípade, že ste napísali
%len časť a ďalej už nechcete, alebo ste hotoví tak sa odtiaľ odpíšte. Bolo by
%však fajn, aby jedu vec robil jeden človek ak celok a zvyšný len kontrolovali
%vypracuva: Jano
\input ../../include/include.tex

\begin{document}

\velkynadpis{Pokračovanie C++}

\textit{Ako správne používať tento študijný text: Milý čitateľ. Chystáš sa
prečítať si kuchárku o základoch programovacieho jazyka C++. Najdôležitejšie
pri učení nového programovacieho jazyka je poriadne si ho precvičiť. Preto
počas čítania narazíš na niekoľko úloh, ktoré ti odporúčame vyriešiť a
naprogramovať.  Má to dopomôcť k tomu, aby si všetko lepšie zapamätal a dostal
do krvi.}

\textit{Tento text je písaný aj pre ľudí, ktorí nikdy program nevideli. Pokiaľ
už máš nejaké základy z programovania, môžeš kludne preskakovať tie časti,
ktoré ovládaš}

Obsah:
\begin{itemize}
    \item podmienky
    \item cykly 
    \item načítavanie vstupu
    \item polia
    \item testovač
\end{itemize}

\medskip

\kapitola{Podmienky alebo rozhodnutia}

Bez toho aby sa program vedel rozhodovať, by sme ďaleko nezašli. Vo väčších
aplikáciách je dôležité interagovať s užívateľom -- \verb!ak! užívateľ stlačí
toto tlačidlo, spravím toto, \verb!inak!  nerobím nič a podobne. 

V našich programoch môžeme chcieť, aby sa program správal odlišne, keď je v
nejakej premennej párne číslo a odlišne, keď je tam nepárne.

Na rozhodovanie v jazyku C++ slúži príkaz \verb!if! (po slovensky 'ak') so
syntaxou \verb!if (<podmienka>) <príkaz-t>! alebo \verb!if (<podmienka>) <príkaz-t> else <príkaz-f>!
('else' = 'inak'). Jeho správanie je veľmi intuitívne, pokiaľ je
\verb!<podmienka>! pravdivá, vykoná sa \verb!<príkaz-t>!.  Pokiaľ je podmienka
nepravdivá, tak vykoná to čo je za \verb!else!, alebo sa nevykoná nič, ak tam
slovíčko \verb!else! nie je.

Príklady použitia:
\lstlang{cpp}\begin{lstlisting}
int x = 1;

if (x == 7) 
    printf("x je sedem\n");
else
    printf("x nie je sedem\n");

if (x > 0)
    printf("x je kladne cislo\n");

if (x > 0)
    if (x < 2)
        printf("x je jedna\n");
\end{lstlisting}

Pokiaľ chceme vykonať viac príkazov, stačí miesto \verb!<príkaz>! napísať \verb!{ <príkaz-1>; <príkaz-2>; ...  <príkaz-n>; }!
\lstlang{cpp}\begin{lstlisting}
int x = 1;
if (x != 7){
    printf("x nie je sedem\n");
    x = 7;
    printf("preto som zmenil x na sedem\n");
} else {
    printf("x nie je sedem\n");
    printf("takze nemusim nic robit\n");
}
\end{lstlisting}

\medskip

Čo všetko môže byť podmienka?  Podmienkou môže byť ľubovoľný výraz, ktorý má
hodnotu typu bool (v prípade, že nevracia, tak sa pretypuje).  Keď má výraz
hodnotu true, považujeme ho za pravdivý, false je nepravdivý.

Poznáme mnoho operátorov, ktoré vracajú true/false. 

Napríklad \verb'==', \verb'!=', \verb'<=', \verb'>=', \verb'<' a \verb'>' patria medzi binárne
operátory (t.j. dávame ich medzi dva výrazy, napr. \verb!(a + 7) <= (b * 2)!).
Správanie je opäť intuitívne, \verb'a < b' je true vtedy, keď \verb!a! je menej ako
\verb!b!.  \verb'!=' znamená nerovná sa, čiže \verb'a != 0' vráti \verb!true!, keď
\verb!a! nie je 0. Pozor, nesmieme si zameniť \verb'a == b' a \verb'a = b'.  Prvé z
toho je výraz, ktorý vráti \verb!true!, v prípade, že \verb!a! má rovnakú hodnotu
ako \verb!b!. Druhá vec je normálne priradenie, takže do \verb!a! vloží hodnotu \verb!b!-čka,
a následne vráti novú hodnotu \verb!a!. Preto \verb!if (a = 7)! vráti \verb!true!
bez ohľadu na pôvodnú hodnotu \verb!a! a navyše zmení hodnotu \verb!a! na sedem.  Na toto
si treba dávať fakt veľký pozor. 

Potom poznáme aj logické operátory, \verb!&&!(a zároveň), \verb!||!(alebo) a
\verb'!'(nie je pravda, že). Ktoré sa správajú opäť intuitívne.  
\verb!(a == 7) && (b < 3)! je \verb!true! práve vtedy, keď \verb!a! má hodnotu 7 a
\verb!b! je menej ako 3. \verb!(a == 3) || (b < 7)! je pravda, keď platí aspoň jedna
z podmienok \verb!(a == 3)!, \verb!(b < 7)!. \verb'!' robí z 
\verb!false! \verb!true! a z \verb!true! \verb!false!, takže napr. \verb'!(a < 3)' je
ekvivalentné s \verb'a >= 3'.  Zaujímavé je, že \verb!||! a \verb!&&! sa
správajú šetrne, napríklad ak máme \verb!a || b! a vieme, že \verb!a! je true, tak
nemusíme vyhodnocovať \verb!b! ('pravda alebo hocičo' je stále pravda). A v C++ sa
to teda ani vyhodnocovať nebude.  Preto zatiaľ čo \verb!(true && a = 5)! a
\verb!(false || a = 5)!  zmenia hodnotu \verb!a! na 5, \verb!(false && a = 5)! a
\verb!(true || a = 5)! ju nechajú na pokoji (príkaz a = 5 sa nevykoná).

Tiež si netreba mýliť \verb!&&! s \verb!&!(bitwise and) a \verb!||! s
\verb!|!(bitwise or). Dvojité verzie pracujú s bool-mi, teda najprv sa pokúsia
pretypovať operandy na booly a potom vyrobia nový bool. \verb!(10 || 6)! je true.
Jednotité verzie rozbijú číslo na bity, s každým bitom spravia operáciu zvlášť
(0 = false, 1 = true) a výsledné bity poskladajú naspať do čísla. 
\verb!(10 | 6) = (1010_2 | 110_2) = 1110_2 = 14!.  Bitová verzia \verb'!' je \verb'~'.

Treba si dávať pozor na prioritu (precedenciu) operátorov (tak ako násobenie má
prednosť pri sčítaní, aj ostatné operátory majú medzi sebou nejakú hierarchiu).
A aby toho nebolo málo, niektoré veci sa vyhodnocujú sprava doľava a niektoré
zľava doprava. Kompletný prehľad nájdete napríklad na
\url{http://en.cppreference.com/w/cpp/language/operator\_precedenceu} Dôležité
je, že ak si nie ste istí, tak zátvorkujte, a inak sa to správa relatívne
rozumne.

\cvicenie Akú pravdivostnú hodnotu má výraz 
\verb'( (!7 == false) && (0 | (a = 5)) <= 4 )'?

\riesenie \verb'( (!7 == false) && (0 | (a = 5)) <= 4 )' $\rightarrow$
\verb'( ( false == false) && (0 | 5) <= 4 )' $\rightarrow$
\verb'( ( true ) && (5) <= 4 )' $\rightarrow$ \verb'( true && false )' $\rightarrow$ \verb'false'

\kapitola{Cykly alebo opakovanie}

Ako by ste 10 krát vypísali "ahoj"? Spravili by ste to takto?
\lstlang{cpp}\begin{lstlisting}
printf("ahoj\n");
printf("ahoj\n");
printf("ahoj\n");
printf("ahoj\n");
printf("ahoj\n");
printf("ahoj\n");
printf("ahoj\n");
printf("ahoj\n");
printf("ahoj\n");
printf("ahoj\n");
\end{lstlisting}
A čo keby ste chceli vypísať "ahoj" tisíckrát, alebo milionkrát?

Nebojte sa, aj na toto probramátori mysleli, a vymysleli cykly.  Najjednoduchší
cyklus je \verb!while! cyklus. Jeho syntax je 
\verb!while(<podmienka>) <príkaz>! prípadne namiesto \verb!<príkazu>! 
môžeme napísať viacero príkazov zabalených v \verb!{}!.

Tento cyklus funguje veľmi jednoducho - dokým je podmienka splnená, opakuje
príkaz. Vysvetlíme si to na príklade.
\lstlang{cpp}\begin{lstlisting}
int i = 0;
while(i < 10){
    printf("ahoj\n");
    i = i+1;
}
\end{lstlisting}
Najprv nastavíme \verb!i! na 1. Následne budeme vykonávať vnútro \verb!{}! pokiaľ 
\verb!i < 10!. Keďže vo vnútri \verb!{}! zväčšíme \verb!i! vždy o 1, celý cyklus sa zopakuje
10 krát, až kým sa nestane, že \verb!i == 10! a teda \verb'!(i < 10)' a teda cyklus
skončí.  Preto aj tento program vypíše "ahoj" 10 krát. Prerobiť ho, aby vypísal
"ahoj" tisíckrát by nám nerobilo žiaden problém však?

Ešte poznamenajme, že miesto \verb!i = i+1! vieme písať \verb!i += 1! alebo
dokonca \verb!i++!, či \verb!++i!.

\medskip

Vyššie uvedená konštrukcia -- nainicializujeme premennú, pokým premenná nie je
nejaká, robíme niečo je natoľko častá, že jej programátori vymysleli nový
názov. \verb!for(<príkaz1>; <podmienka>; <príkaz2>) <príkaz-telo-cyklu>! najprv
vykoná \verb!<príkaz1>! následne kým platí podmienka, vykonáva
\verb!<príkaz-telo-cyklu>! a \verb!<príkaz2>!.

Vyššie uvedený program teda vieme prepísať na
\lstlang{cpp}\begin{lstlisting}
for(int i = 0; i<10; ++i)
    printf("ahoj\n");
\end{lstlisting}

Samozrejme premennú i, môžeme využiť aj v tele cyklu, nasledovný program vypíše
čísla 0 až 7, každé na zvlášť riadku.
\lstlang{cpp}\begin{lstlisting}
for(int i = 0; i<8; ++i)
    printf("%d\n", i);
\end{lstlisting}

\cvicenie Napíšte program, ktorý vypíše prvých 50 nepárnych čísel (1, 3,
5, \dots) Zvyšok po delení dvoma (paritu) vieme zistiť napríklad operátorom \verb!%!.
\verb!a % b! vráti zvyšok \verb!a! po delení \verb!b!. Napríklad \verb!47 % 10 == 7!, \verb!10 % 2 == 0!.
Pozor treba dávať na záporné čísla, \verb!-13 % 5 == -3!.

\riesenie Predvedieme si tri spôsoby. 
\lstlang{cpp}\begin{lstlisting}
for(int i = 0; i<100; ++i)
    if (i%2 == 1)
        printf("%d\n", i);

for(int i = 0; i<50; ++i)
    printf("%d\n", 2*i + 1);

for(int i = 1; i<100; i+=2)
    printf("%d\n", i);
\end{lstlisting}

Pri cykloch si treba dávať pozor, keďže sa môžu zacykliť a potom váš program
bude bežať donekonečna (kým ho nezabijete).  Príklady nekonečných cyklov sú
napr. \verb!while(true);! \verb!for(int i = 10; i>0; ++i);! ale aj
\verb!while(a<7);! (lebo premenná a sa v cykle nemení).

\medskip

Cykly a podmienky sa samozrejme dajú do seba ľubovoľne veľa krát vkladať, napríklad ako v nasledujúcom
nezmyselnom programe.

\lstlang{cpp}\begin{lstlisting}
int a = 7;
while(a<70) {
    if (a == 2) {
        a = 31;
    } else {
        for (int i = 0; i<a; ++i) {
            while(i<a) ++i;
        }
    }
}
\end{lstlisting}


\kapitola{Načítavanie vstupu}

Doteraz všetky programy boli relatívne nanič, pretože pri každom spustení
vypisovali to isté.  Namiesto programov typu "vypíš prvých 10 nepárnych čísel"
by sme radi mali "načítaj číslo $n$ a vypíš prvých $n$ nepárnych čísel".

Takže potrebujeme vedieť načítavať vstup.

Na načítavanie sa dá použiť funkcia \verb!scanf()!, ktorá sa syntaxou veľmi
podobá \verb!printf()!. Prvý parameter je opäť formátovací reťazec, ale ďalšie
už nie sú výrazy, ale smerníky na premenné. Teda ak chceme načítať jedno číslo
do premennej a, napíšeme \verb!scanf("%d", &a);!

Prečo tam je \verb!&a! miesto \verb!a!? Dôvod je veľmi jednoduchý, \verb!a! je
výraz, ktorého hodnota je hodnota \verb!a!, napríklad číslo 7.  A zavolaním
\verb!scanf("%d", 7);! ťažko budeme vedieť, kam máme uložiť načítanú hodnotu.
(Vieme ukladať len do premenných, nie do hodnnôt). Keď však napíšeme \verb!&a!,
tak výsledkom výrazu nebude hodnota \verb!a!, ale miesto, kde je \verb!a!
uložené v pamäti počítača.  Funkcia teda načíta číslo a uloží ho na to miesto v
pamäti, kde bola pred tým hodnota \verb!a!.

Vyskúšajte si skompilovať a spustiť tento program:
\lstlang{cpp}\begin{lstlisting}
#include<cstdio>
int main(){
    int a, b;
    scanf("%d %d", &a, &b);
    printf("%d\n", a+b);
}
\end{lstlisting}

Po spustení, program čaká, kým mu napíšete dve čísla napr. \verb"105 21". Po
zadaní čísel musíte stlačiť enter, inak sa k programu nedostanú. 

Pokiaľ sa vám \verb!scanf! nepáči, dajú sa použiť streamy.
\lstlang{cpp}\begin{lstlisting}
#include<iostream>
using namespace std;
int main(){
    int a, b;
    cin >> a >> b;
    cout << a + b << endl;
}
\end{lstlisting}

Časom sa naučíme načítavať aj iné veci ako čísla, aby sme to však zvládali, tak
musíme tie iné veci spoznať.

\kapitola{Polia}

Sú situácie, kde si jednoducho neporadíme s obmedzeným množstvom premenných.
(Vedeli by ste načítať $n$ čísel zo vstupu a následne ich vypísať v opačnom
poradí?) To sa s obmedzeným množstvom číselných premenných nedá, pretože ak je
na vstupe viac čísel, ako máme premenných, tak si ich nedokážeme zapamätať na
neskorší výpis. 

Aj keby sme mali napríklad na vstupe 20 čísel, tak je veľmi nepraktické vyrobiť
si premenné \verb!a0! až \verb!a19! do ktorých si vstup načítame.

Preto existujú polia. Pole je akokeby skupina premenných rovnakého typu
naukladaných za sebou.  V súvislosti s nimi používame hrnaté závtorky
\verb![]!. Pole intov dĺžky 20 vyrobíme napríklad takto \verb!int pole[20];!
Tento príkaz nám vyrobí 20 \verb!int!ových premenných ku ktorým vieme pristupovať
\verb!pole[0]!, \verb!pole[1]!, \verb!pole[2]!, \dots \verb!pole[19]!.

Pozrite si príklad použitia v nasledovnom programe:
\lstlang{cpp}\begin{lstlisting}
#include<cstdio>
int main(){
    int a[47];
    a[0] = 0;
    a[1] = 1;
    for(int i = 2; i<30; ++i){
        a[i] = a[i-1] + a[i-2];
    }
    for(int i = 0; i<30; ++i)
        printf("%d ", a[i]);
    printf("\n");
}
\end{lstlisting}

Na čo si treba dávať pozor?

Keď pristupujeme mimo poľa, napríklad \verb!int b[40];! \verb!b[53] = 7;!
\verb!b[-1] = 3;! \verb!int a = b[40];! \dots dejú sa zlé veci. Niekedy program
hneď spadne a vyhlási chybu, inokedy pokračuje ďalej a robí čudné veci,
premenným sa záhadne zmení hodnota vykonávajú sa \verb!if!y, ktoré by nemali
atď.  Preto si vždy dávajte veľký pozor, keď používate polia, aby ste
nepristupovali mimo nich.  Pozor treba dávať aj na to, že \verb!int b[40];!
tvoria premenné \verb!b[0]! až \verb!b[39]!  (teda ich je 40, ale premenná
\verb!b[40]! tam nie je).

Pri deklarácii poľa musí byť v \verb![]! konštanta.  \textbf{Nesmiete} spraviť
\verb!int a = 7; int b[a];! hoci vám to kompilátor pravdepodobne dovolí, môžu
sa diať zlé a nepredvídateľné veci.  Pokiaľ chcete pole so závislou veľkosťou,
použite \verb!int *b = new b[a];! ale reálne toto nebudete potrebovať. V
drvivej väčšine úloh máte zadané obmedzenia, napríklad "na vstupe je číslo $n$,
$1\leq n\leq 1000$". Vtedy, ak budete potrebovať pole veľkosti $n$, tak si
môžete natvrdo vyrobiť pole veľkosti 1000 a netrápiť sa.

Odporúčame však vyrábať troška väčšie polia, napríklad keď $n<10000$ tak pole
by mohlo mať veľkosť 10047. Menej ako 1\% nárast nič nepokazí, a trocha pomôže,
ak náhodou siahame o pár políčok ďalej, ako by sme čakali. 

Pamäť počítača má prekvapivo obmedzenú veľkosť. Jeden int zaberá 4 bajty, takže
pole \verb!int velkepole[1000000];! zaberá približne 1MB. Bežne na sútažiach
máte pamäťový limit obmedzený na 64 MB, 256 MB, alebo 1024MB, tak na to treba
myslieť.

Navyše na bežných počítačoch je limitovaná veľkosť lokálnych premenných (tie sú
vo vnútri \verb!main()! alebo iných funkcií, na 8MB, takže to tiež môže robiť
problémy, keď si doma budete spúšťať programy s veľkými pamäťovými nárokmi (na
našom testovači by tento limit nemal byť).

\medskip

Pole je v podstate zasa len premenná a z premenných vieme robiť polia, takže aj
z polí vieme robiť polia. Pole polí (dvojrozmerné pole) vieme vyrobiť napríklad
takto \verb!int A[1000][1000];! Takto vieme ísť aj ďalej, len treba zasa dávať
pozor na pamäť \verb!int A[100][100][100][10][10];! zaberá cca 400MB.

\kapitola{Zoznámte sa s testovačom}

Teraz už by ste mali vedieť dosť na to, aby ste vyrišili tieto úlohy:


%\cvicenie 

%\riesenie 

%\lstlang{cpp}\begin{lstlisting}
%\end{lstlisting}




%\cvicenie 

%\riesenie 

%\lstlang{cpp}\begin{lstlisting}
%#include<cstdio>
%int main(){
%    ll a = 1000000*1000000;    
%    printf("d = %lld\n",d);
%}
%\end{lstlisting}

\end{document}


