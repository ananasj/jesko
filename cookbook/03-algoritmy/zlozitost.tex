%Tu si môžete zaznačiť, že pracujete na danej veci. V prípade, že ste napísali len časť a ďalej už
%nechcete, alebo ste hotoví tak sa odtiaľ odpíšte. Bolo by však fajn, aby jedu vec robil jeden
%človek ak celok a zvyšný len kontrolovali
%vypracuva: Zaba
%
%pre inspiraciu si precitajte toto: http://ksp.mff.cuni.cz/tasks/25/cook1.html
\input ../../include/include.tex

\begin{document}

\velkynadpis{Zložitosť}

Obsah:
\begin{itemize}
    \item cieľ a zámer vyjadrovania zložitosti
    \item $O$-notácia
    \item očakávaná zložitosť od veľkosti vstupu
    \item príklady na notáciu
\end{itemize}

Keď riešime nejaké problémy, môže sa stať, že vymyslím viacero riešení. Ako však povedať, ktoré je
lepšie a vôbec čo to znamená byť lepším. O tom všetkom sa porozprávame práve v tejto kapitole.

Dôležitým aspektom je rýchlosť daného algoritmu. To znamená, ako rýchlo sa pre nejaký vstup doráta k
výsledku. Nechceme predsa čakať na výsledok niekoľko hodín, ak sa dá vyrátať v priebehu sekúnd.
Poďme sa teda zamýšľať nad takzvanou \texttt{časovou zložitosťou} algoritmov.

Hneď na začiatku sa však stretávame s množstvom problémov. Prvým je to, že nechceme porovnávať,
skutočný čas výpočtu. Ten je totiž príliš ovplyvnený počítačom, na ktorom daný program spúšťame.
Nečakáme predsa, že na počítači našej babky bude bežať rovnaký algoritmus takisto rýchlo ako na
počítači v Googli. Musíme sa teda snažiť oddeliť časovú zložitosť algoritmu od výpočtu na skutočnom
počítači. Bolo by teda fajn, aby sme vedeli odhadnúť  rýchlosť programu z obyčajného pohľadu naň.

Našťastie, to však vôbec nie je také ťažké. Skúste si to sami. Tu vidíte tri programy, ktoré oba
rátajú tú istú vec -- súčet prvých $n$ čísiel.

%\listing{program1.cpp}

%\listing{program2.cpp}

%\listing{program3.cpp}

Myslím, že aj vám je na prvý pohľad jasné, ktorý z týchto programov je najrýchlejší a naopak, ktorý je
najhorší. Vynásobiť dve čísla je totiž určite jednoduchšie a rýchlejšie, ako pričítavať po
jednotkách k výsledku.

Keď sa pozrieme na druhý a tretí program, vieme veľmi ľahko povedať, koľko sčítaní, ktorý z nich
potrebuje. Druhý spraví $n$ sčítaní, tretí spraví $n(n-1)/2$ sčítaní. A dá sa predpokladať, že
sčítanie trvá na konkrétnom počítači vždy rovnako dlho. Zrazu teda vieme bez ohľadu na rýchlosť
konkrétneho počítaču povedať, ktorý je rýchlejší. Jeden totiž spraví o dosť viac sčítaní ako druhý.
Operáciu sčítanie potom nazvime \texttt{elementárnou operáciou} -- to znamená, že táto operácia trvá
počítaču krátky čas, ktorý je naviac nemenný. Aj keď je to prekvapivé, počítaču trvá rovnako dlho
vyrátať $1+1$ ako $1574123+478413$.

Nezastavme sa však len pri sčítaní. Je aj viac operácií, ktoré vie počítač vykonávať veľmi rýchlo, a
teda môžu byť považované za elementárne. Väčšinou do nich patria všetky aritmetické operácie ako
sčítanie, odčítanie, násobenie, delenie, zvyšok po delení, všetky logické operácie -- and, or, xor
\dots Ďalej tam patrí aj príkaz priradenia, alebo náhľad do poľa či inicializovanie jednej
premennej, načítanie alebo výpis jedného znaku.

Reálne je to bohužiaľ trošku komplikovanejšie, napríklad zvyšok po delení je náročnejší ako bitový
and. Ale nám stačí takáto abstrakcia. Pri veľkých vstupoch to málokedy hrá hlavnú úlohu a dodáva to
oveľa lepšiu možnosť sa vyjadrovať. A o tom, ako je to v skutočnosti sa môžeme pobaviť neskôr.

Vieme teda už počítať elementárne operácie programov. Zrazu je nám úplne jasné, prečo je prvý
program najrýchlejší. Stačí mu predsa spraviť jednu jedinú operáciu. A aj u zvyšných dvoch je jasné,
ktorý je lepší. Tu však problémy nekončia. Predstavme si, že máme dva programy, ktoré na vstupe
dostanú jedno číslo $n$ a niečo s ním rátajú. Prvý spravý $20n$ operácií, druhý $n^2$. Ktorý z nich
je rýchlejší? Pre $n<20$ je to program druhý, potom už vždy vyhrá program prvý. Intuitívne je teda
prvý program lepší, veď na väčšine vstupov je rýchlejší. My si však túto intuíciu poriadne
definujeme.

\nadpis{Asymptotická zložitosť}

Uvedomme si, že počet elementárnych operácií programu sa dá vyjadriť ako
funkcia od premenných na vstupe. Ak máme na vstupe číslo $n$, bude to funkcia tohoto čísla. A my
chceme nejak porovnávať tieto funkcie. Dobrá miera toho, je schopnosť rásť. Čím rýchlejšie funkcia
rastie, tým viac operácií bude treba. Takisto nás poväčšine nezaujímajú malé vstupy, ale také, na
ktorých zrátanie treba milióny operácií. A samozrejme nechceme byť zbytočne presný. Však $1000000$ a
$1000042$ je skoro to isté. Hore dole $42$ operácií, také niečo môžeme zanedbať.

A na toto všetko je vytvorená takzvaná $O$-notácia a asymptotické určovanie zložitosti.
Hneď vás varujem. Nasledujúce odseky budú obsahovať nejaké formálne definície aj niektoré neformálne
tvrdenia. Ono je to celé veľmi intuitívne a keď s tým začnete pracovať, rýchlo si na to zvyknete.
Môže to byť však zo začiatku trochu mätúce, pokúsim sa vám to však podať čo najlepšie.

%TODO dorobiť obrázky funkcií

Začnime s príkladom. Majme program, ktorý vykoná $f(n) = n^2/2 + 5n + 4$. Prvé pravidlo znie,
konštanty zanedbáme. To či spravíme operácií $n$ alebo $5n$ je skoro jedno, zdvihne sa to len o
konštantu, ktorá nezávisí od veľkosti vstupu. Zrazu máme teda $f'(n) = n^2 + n$. Druhé pravidlo zase
hovorí o tom, že nás zaujíma len najväčší člen funkcie, teda ten, čo najrýchlejšie rastie. V tomto
prípade $n^2$ rastie oveľa rýchlejšie ako $n$. Tým pádom dostaneme výslednú funkciu $f''(n) = n^2$.
A to je náš odhad na to, koľko operácii spraví náš program.

Možno sa to zdá hrubé, koľko vecí sme zanedbali, ale ono to vôbec nie je tak. Tých zanedbaných vecí
je najviac konštantne viac ako to čo nám zostalo. A to nám stačí. Samozrejme takýto spôsob škrtania
je škaredý a tie dve pravidlá sú len pomôcky. Potrebujeme jasnú definíciu a tou je už spomínaná
$O$-notácia.

\textbf{Definícia:} Majme dve funkcie $f$ a $g$. Hovoríme, že $f \in O(g)$ ak existuje $n_0$ a $c>0$
také, že pre všetky $n \geq n_0$ platí $|f(n)|\leq c|g(n)|$.

\textbf{A slovami:} Hovoríme, že funkcia $f$ patrí do $O(g)$ (všimnite si, že $O(g)$ je množina
všetkýh funkií, pre ktoré platí daná vlastnosť) ak existujú také konštanty $n_0$ a $c$, že pre dosť veľké $n$ (lebo $n \geq n_0$) je
$c|g(n)|$ (absolútna hodnota) väčšia ako $f(n)$.

Teraz už ľahko určíme, že naozaj naše $f(n) = n^2/2 + 5n +4 = O(n^2)$ (tento zápis síce nie je úplne
korektný, lebo $O(n^2)$ je množina, napriek tomu sa často používa). Stačí, ak si zvolíme $n_0=10$ a
$c=2$. Ľahko potom nahliadneme, že $n^2/2 + 5n + 4 \leq 2n^2$, lebo toto je vlastne $5n + 4 \leq
3n^2/2$. Táto nerovnosť pre $n=10$ platí a ďalej bude už len masívne narastať.

Tiež si všimnite, že napríklad $n^3 \notin O(n^2)$. Ak si zvolíte totiž ľubovoľné $c$, tak $n^3$ je
väčšie od $cn^2$ pre všetky $n\geq c$. Naopak však platí, že $n^2 \in O(n^3)$, lebo to vyhovuje
definícii pre $c=n_0=1$.

\nadpis{Najhorší prípad a dolná hranica}

Samozrejme, nikto neočakáva, že hneď pochopíte čo to vlastne znamená. Postupne sa však s
$O$-notáciou budete stretávať neustále. Práve v nej sa totiž neustále vyjadruje časová aj pamäťová
zložitosť všetkých algoritmov. My si ukážeme pár funkií, s ktorými sa stretnete najčastejšie.

Ešte sme si však nepovedali, na ktorom vstupe rátame zložitosť. Môže sa totiž stať, že niekde bude
náš program potrebovať len $O(n)$ operácií a inde zase $O(n^2)$. V tomto prípade samozrejme berieme
najhorší možný príklad a zložitosť výpočtu na ňom prehlásime za zložitosť riešenia.

Ďalšia vec je, že my informatici sme tak trochu flákači. Vyjadrovanie sa v $O$-notácii je totiž
mierne nepresné. Ako sme totiž ukazovali, tak $n^2 = O(n^3)$. Ak máme teda algoritmus so zložitosťou
$f(n)$, nič nám nebráni povedať, že to je vlastne $O(n^3)$. Nás však zaujíma, takpovediac najmenšia
dolná hranica, v tomto prípade $n^2$. Takúto zložitosť by ste mali udávať aj všade, kde po vás
požadujú zložitosť.

%TODO tabuľka s častými zložitosťami

\end{document}
