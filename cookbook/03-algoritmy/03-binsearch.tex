%Tu si môžete zaznačiť, že pracujete na danej veci. V prípade, že ste napísali len časť a ďalej už
%nechcete, alebo ste hotoví tak sa odtiaľ odpíšte. Bolo by však fajn, aby jedu vec robil jeden
%človek ak celok a zvyšný len kontrolovali
%vypracuva:Zaba 

%gram. oprava Maja


\input ../../include/include.tex

\begin{document}

\velkynadpis{Binárne vyhľadávanie}

Obsah:
\begin{itemize}
    \item cieľ tejto časti
    \item princíp binárneho vyhľadávania
    \item binárne vyhľadávanie odpovedí
    \item binárne vyhľadávanie bez jednej hranice (+1, +2, +4 \dots)
    \item ternárne vyhľadávanie
\end{itemize}

Už sa vám určite stalo, že ste zobrali do ruky učebnicu dejepisu a snažili ste sa nájsť tú správnu stranu, na
ktorej sú poznámky k zajtrajšej písomke. V takomto prípade určite nie je času nazvýš a treba nájsť
tú správnu stranu čo najrýchlejšie (keby ste nevedeli, tá správna je strana $47$).

Samozrejme, môžeme v nej ísť od začiatku, postupne sa pozrieť na stranu $1$, $2$ \dots až
konečne dorazíme na stranu $47$. Takto však prezrieme veľmi veľa stránok, pričom je nám jasné, že
mnohé z nich by sme mohli preskočiť.

Oveľa pravdepodobnejšie je, že zoberieme našu učebnicu a otvoríme ju na náhodnej strane. Nachádzame
číslo $34$. Prichádza dôležitá úvaha: keďže sú strany v knihe radené postupne od $1$ a čísla na
seba nadväzujú, je nám jasné, že strana $47$ bude niekde za touto stranou. Odrazu sa teda nemusíme
pozerať na množstvo strán, ktoré idú pred $34$ a ušetrili sme si oproti predchádzajúcemu postupu
$33$ pozretí.

Teraz by sme už mohli skončiť a stranu $47$ ručne dohľadať alebo môžeme znova otvoriť knihu na
náhodnej strane medzi stranou $34$ a koncom knihy. A presne tento postup sa volá binárne
vyhľadávanie.

\medskip

Zadefinujme si úlohu trošku všeobecnejšie a trošku programátorskejšie. Majme pole, kde je uložených
$n$ celých čísel a naviac tieto čísla idú v poradí od najmenšieho po najväčšie. Našou úlohou je
zistiť, či sa v tomto poli nachádza číslo $x$ a pozerať pritom čo najmenej do poľa. V
prípade, že by sme sa pozreli postupne na všetky prvky a porovnávali ich s $x$, urobili by sme
$O(n)$ pohľadov do poľa. Našim cieľom bude toto číslo zmenšiť.

Nech dvojica $(z, k)$ je dvojica indexov, ktoré označujú "aktívnu" časť poľa -- interval, kde sa ešte
prvok $x$ môže nachádzať. O všetkom mimo týchto indexov vieme, že sa tam číslo $x$ určite nebude
nachádzať. Naviac použijeme polootvorený interval, teda pozícia $z$ je ešte aktívna, ale na pozícii
$k$ sa už číslo $x$ vyskytovať nemôže. To znamená, že ak sa nám podarí náš interval zredukovať tak,
že $z+1 = k$ ostane nám jediné aktívne číslo na pozícii $z$. Ak sa toto číslo rovná $x$, $x$ sa v
poli nachádza, v opačnom prípade žiadne číslo z poľa nie je $x$.

Zvoľme si teraz stred tohto intervalu, teda index $s = \frac{z+k}{2}$. Pozrime sa na číslo na
$s$-tom mieste. Ak je toto číslo menšie alebo rovné ako $x$, znamená to, že všetky indexy menšie ako
$s$ musia obsahovať čísla menšie ako $x$, lebo pole je usporiadané od najmenších prvkov. To znamená,
že sa náš aktívny interval zredukuje na $(s, k)$.

V opačnom prípade, keď je prvok na $s$-tej pozícii väčší ako $x$, tak sa prvok $x$ nemôže nachádzať
napravo od $s$, lebo tam sú čísla ešte väčšie. Náš aktívny úsek sa zmenší na $(z, s)$. Rozmyslite
si, že naša polootvorenosť tohto intervalu sa týmto nepokazila.

Koľkokrát sa pri takomto prístupe pozrieme do poľa, ak začíname intervalom $(0, n)$? Zakaždým sme prvok $s$ vybrali ako stred
intervalu $(z, k)$ a zakaždým sme zahodili jednu polovicu tohto intervalu. To znamená, že na
začiatku sme mali $n$ aktívnych prvkov, potom $\frac{n}{2}$, $\frac{n}{4}$, $\frac{n}{8}$ \dots
Ak by bolo $n$ mocnina dvojky, teda $n = 2^k$, tak by sme mohli tento interval deliť na dve polovice najviac
$k$ krát a potom by sme dostali len jeden samostaný prvok a teda by sme vedeli odpoveď. No a správna
funkcia, ktorá nám z čísla $2^k$ spraví číslo $k$ je logaritmus, takže sa do poľa pozrieme rádovo
$O(\log n)$ krát.

Tu je kratučký program, ktorý robí práve vyššie spomínanú vec.

\listing{binarysearch.cpp}

Možno tomu neveríte, ale aj takáto jednoduchá myšlienka sa dá skryť do mnohých úloh a stáva sa
mocnou zbraňou každého programátora. Môžete si to hneď aj vyskúšať na úlohe \textbf{najdi}

\nadpis{Binárne vyhľadávanie výsledku}

Tu však nekončíme s naším rozoberaním binárneho vyhľadávania. To sa totiž pri riešení úloh používa
aj inak ako na hľadanie prvkov v poli alebo na hľadanie vecí v dopredu známych množinách prvkov.
Niekedy sa totiž môže stať, že vyrátať nejakú odpoveď je veľmi ťažké, ale overiť si, či je nejaká
odpoveď prípustná, je oveľa ľahšie.

Zoberme si príklad \textbf{rozdavanie}. Máme $p$ cukríkov a všetky ich chceme rozdeliť medzi $n$
detí. O každom z nich vieme, koľko najviac cukríkov smie dostať (číslo $a_i$) a určite mu nedáme
viac. Chceme však byť aspoň trochu spravodliví a rozdeliť cukríky čo najrovnomernejšie. Hľadáme teda
čo najmenšie číslo $x$ také, že rozdáme všetky cukríky a žiadne dieťa nedostane viac ako $x$ cukríkov.

Zistiť hodnotu $x$ priamo je obtiažne. Zvoľme si však pevne, že $x = 42$. Chceme overiť, či $x$ je
vyhovujúci počet cukríkov alebo nie. Zistíme teda, koľko cukríkov rozdáme, keď máme takéto $x$.
Pozrieme sa na každé dieťa. Ak si dieťa zaslúži menej ako $42$ cukríkov, dáme mu všetky, ktoré si
zaslúži, inak mu dáme práve $42$ cukríkov, takže žiadne dieťa nedostane viac. Nech $r$ je počet cukríkov,
ktoré sme takto rozdali. Ak $r < p$, tak číslo $x$ je príliš malé, lebo nerozdáme všetky cukríky. V
opačnom prípade rozdáme všetky cukríky, môže sa však stať, že existuje aj menšie $x$, ktoré
vyhovuje. Avšak $42$ je určite dobrý počet cukríkov.

Vieme teraz na to napasovať binárne vyhľadávanie? Samozrejme, že áno :). Budeme hľadať najmenšie $x$.
Dôležitá vlastnosť, ktorú musí mať číslo $x$, je monotónnosť. To znamená, že zo začiatku sú všetky
$x$ nevhodné a potom príde zlom a všetky väčšie čísla už vhodné sú. Ak nájdeme daný zlom, tak
číslo na ňom je hľadaná odpoveď, lebo je to najmenšie z vhodných čísel.

Keď sa zamyslíme, tak naša odpoveď túto vlastnosť má. $x = 0$ zjavne odpoveď nebude, lebo nepoužijeme
žiaden cukrík. A ak nejaké $x$ je správna odpoveď, tak aj všetky ostatné sedieť budú, lebo použijeme
aspoň toľko cukríkov.

Interval, na ktorom budem hľadať odpoveď bude $(0, maxa)$, kde $maxa$ je najväčšie číslo $a_i$.
Uvedomte si, že väčšia byť odpoveď nemôže, lebo viac cukríkov nerozdám. Na tomto intervale
budeme binárne vyhľadávať odpoveď, overíme ju vždy spomínaným spôsobom a teda porátaním čísla $r$.
Na konci nám vyjde jedno číslo -- správna odpoveď.

\medskip

Teraz už vieš o binárnom vyhľadávaní skoro všetko :). Ostáva si ho už len usilovne precvičiť aj na
týchto úlohách: \textbf{najdi2}, \textbf{obrazky}, \textbf{susenie} či \textbf{faktorial}.

\end{document}
